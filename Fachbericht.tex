\documentclass{fhnwreport}
\usepackage[ngerman]{babel}
\usepackage[T1]{fontenc}
\usepackage[utf8]{inputenc}
\usepackage[T1]{fontenc}
\usepackage{subfigure}
\usepackage{tikz}
\usepackage{amsmath}
\numberwithin{equation}{section}
\usetikzlibrary{arrows}
\usepackage{lmodern}   %Type1-Schriftart für nicht-englische Texte
\usepackage{float}
\usepackage{color}

\title{
  \textsc{Fachbericht}\\[2ex]
  \textsc{FS18 - Pro4E - Team 5}  }
\bibliographystyle{fhnwreport/IEEEtran}  

\begin{document}
\maketitle

\textsc{%
\begin{tabbing}
Auftraggeber: \hspace{4em} \= H. Gysin \\[2ex]
\> J. Kalbermatter \\ [2ex]\\
Betreuer:  \>  M. Meier\\[2ex]
\> A. Gertiser \\[2ex]
\> R. Dubach \\[2ex]
\> B. Domenghino\\[2ex]
\> P. Schleuniger \\[2ex]\\
Projektleitung: \> Simon Zoller\\[2ex]\\
Teammitglieder: \> Severin Hunziker\\[2ex]
\> Mischa Knupfer\\[2ex]
\> Lukas Loosli\\[2ex]
\> Josha Giambonini\\[2ex]
\> Elias von Däniken\\[2ex]
\> Gianluca Picciola\\[2ex]\\
Studiengang: \> Elektro- und Informationstechnik\\[2ex]
\end{tabbing}
}


\hbox{}
\clearpage


%%%%%%%%%%%%%%%%%%%Abstract%%%%%%%%%%%%%%%%%%%%%%%

\section*{Abstract}\label{sec:abstract}
Aimlessly wandering around a museum because of too little knowledge is one of the main negative reactions to a museum visit. That’s why no one can remember an individual work of art. To counteract this, an audio guide called Dōjō was designed to provide visitors with key information about a work of art so they can enjoy their visit more. This project should design the circuits for this Dōjō, which electronically recognises the artwork the visitor is standing in front of. Audio information stored on an internal SD-Card over a Class-D Amplifier is provided through an integrated bone sound sensor produced by adafruit. Buttons for audio control have also been integrated. In addition, the Dōjō allows the visitor to \glqq like\grqq an artwork by simply pressing a button. Information about these \glqq liked\grqq works are made available to the visitors either by email or a hardcopy is provided at the end of their visit. Each work of art has a Bluetooth Low Energy (BLE) Beacon that continuously sends its ID to the Dōjō. The Dōjō’s internal microcontroller (nrf52832) then scans for this BLE-Signals and plays the file that belongs to the ID with the strongest signal. The Dōjō can identify beacons from 0 to 40 meters. The inductive rechargeable battery has a capacity of 800mAh and can provide energy for about 6 hours when the Dojo is playing audiofiles and be recharged in normal charging mode within 13 hours. To protect the battery from damage, a deep discharge protection turns off the Dōjō when the battery is under 3V of its nominal voltage.\\[0.25cm]
Key Words: audio guide, Bluetooth low energy, inductive charge, bone sound sensor
\newpage

%%%%%%%%%%%%%%%%%%%%%%%%%%%%%%%%%%%%%%%%%%%%%%%%%%%
\setcounter{tocdepth}{2} 
\tableofcontents
\newpage

%Input Files
%%%%%%%%%%%%%%%%%%%%%%%%%%%%%%%%%%%%%%%%%%%%%%%%%%%%%%%%
\section{Einleitung}\label{sec:einleitung}

Museen bieten die Möglichkeit unterschiedlichste Ausstellungsobjekte unter einem Dach zu betrachten. Die Art der ausgestellten Kunst ist hier von Austellung zu Austellung unterschiedlich, was jedoch bliebt ist die kreative Wahrnehmung der Besucher. Wohl nirgends kann man so gut in seinen eigenen Gedanken versinken und sich Gedanken über ein Ausstellungsobjekt machen wie bei einem Museum. Um Besucher anzulocken, sind Museen auf Innovation angewiesen, welche zum einen die Übergabe von Informationen möglichst Benutzerfreundlich gestaltet aber auch eine angenehme Ambiance schaffen. Hierbei kommt auch vermehrt der Einsatz von Smartphones zum Zuge, wobei die Problematik darin besteht, dass man der Aussenwelt gefährlich nahe kommt und dadurch abgelenkt wird. Um diesem Problem entgegenzuwirken, wird ein von Frau J. Kalbermatter designter Audioguide namens Dōjō entwickelt, welcher Kunstobjekte drahtlos erkennen und darüber gespeicherte Informationen via Körperschalltechnik zum Museumsbesucher bringen kann. Durch eine Vielzahl von weiteren Funktionen ist ein neuer Informationsaustausch an den Museumsbesucher möglich, und Ablenkungen zur Aussenwelt sind trotz modernster Technik in weiter Ferne.
 
Ziel im Projekt 4, Studiengang Elektro- und Informationstechnik an der Fachhochschule Nordwestschweiz, war es das funktionelle Konzept von Frau Kalbermatter durch die Verwendung von elektrotechnischen Bauteilen zu realisieren. Dazu wurde die drahtlose Erkennung der Kunstobjekte mittels Bluetooth Low Energy (BLE) Beacons erreicht. Genannte Beacons müssen in unmittelbare Nähe der Kunstobjekte angebracht sein. Die Informationen zu den Kunstobjekten wurden als Audiofiles auf einer herausnehmbaren SD-Karte gespeichert und werden zum Abspielen via PWM-Ausgang des Mikrokontrollers (nRF52832) über einen Klasse D Verstärker auf den Knochenschallaktor gegeben. Tasten für die Wiedergabekontrolle (Start, Stopp, Lauter, Leiser) wurden implementiert sowie die erwähnte \glqq Like\grqq -Taste. Ausserdem verfügt der Dōjō über einen Li-Ionen-Akku, welcher induktiv geladen wird. Damit der Dōjō gänzlich drahtlos bleibt, erfolgen Datendownload und Konfiguration ebenso über Bluetooth. Der integrierte Mikrokontroller beinhaltet die Software und übernimmt somit die Erkennung, Ansteuerung und Koordination der Hardware.
 
Es wurde ein Prototyp der Elektronik realisiert. Er kann zu XXX verschiedenen Sprachen XXX Stunden Audioausgabe speichern. Die Ansteuerung der Audiofiles erfolgt über Bluetooth-Beacons, welche bis zu einer Distanz von XXX m erkennt werden. Die eingebaute «Like»-Taste ermöglicht, favorisierte Kunstobjekte zu vermerken und die dazugehörigen Informationen am Ende des Besuches digital oder in Form einer Broschüre beim Ausgang als Erinnerung mitzunehmen. Ausserdem besitzt der Dōjō einen integrierten Akku mit einer Kapazität von XXX mAh, welcher bei pausenloser Audioausgabe genug Energie für XXX Stunden liefert. Die Induktionsladung lädt den Akku zu 100\% innert XXX Stunden. Ausserdem sorgt ein Tiefenentladungsschutz dafür, dass der Dōjō bei XXX\% Akkuladestand ausgeschaltet wird um den Akku vor Schäden zu bewahren.
 
Der nachfolgende Bericht umfasst drei Hauptbereiche. Der erste Bereich (Kapitel 2) umfasst das Gesamtkonzept, welcher die gesamte Anwendung auslegt. Die nachfolgenden zwei Hauptbereiche sind in Hardware (Kapitel 3) und Software (Kapitel 4) gegliedert. Die Hardware teil sich wiederum in die Themengebiete Energieübertragung (Kapitel 3.1), Energiespeicherung (Kapitel 3.2) und Audioausgabe über den Knochenschallaktor (Kapitel 3.5) auf. Die Software beinhaltet die Unterbereiche der State Machine (Kapitel 4.1), Bluetooth (Kapitel 4.3), sowie die gesamte Programmstruktur der SD-Karte (Kapitel 4.4) und Audioausgabe über PWM (Kapitel 4.5). In Kapitel 5 befindet sich die Validierung.\\

\newpage
\section{Gesamtkonzept}\label{sec:gesamtkonzept}


\subsection{Funktionsweise} \label{sec:funktionsweise}
Der Dōjō ist eine Art Audioguide, welcher für Museen designt wurde und diverse Funktionen beinhaltet. Abbildung \ref{fig:Funktion Dojo} zeigt den von der Auftraggeberin designten Prototypen. Einer der grössten Unterschiede zu einem herkömmlichen Audioguide ist die Sprachausgabe mittels einem Knochenschallaktor und nicht wie gewöhlich mit einem Lautsprecher. Eine weitere Eigenheit ist der integrierte {\glqq Like\grqq}-Button, mit dem man die entsprechenden Ausstellungsstücke {\glqq liken\grqq} kann. Dadurch lässt sich für jeden Besucher individuell eine persönliche Kunstobjektliste zusammenstellen, die nur noch die entsprechenden Objekte mit einem {\glqq Like\grqq} enthält. Das ermöglicht es, am Ende des Besuchs jeder Person eine Liste mit den jeweiligen Informationen zum Ausstellungsobjekt auszuhändigen. Ansonsten verfügt der Dōjō über die gleichen Funktionen, die man von einem üblichen Audioguide erwarten würde, wie z.B. der Audiowiedergabe, dem Haltemodus oder der Lautstärkeregelung.

\begin{figure}[H]
	\begin{center}
		\includegraphics[width=120mm]{data/Dojo.png}
		\caption[Dōjō als Modell]{Dōjō als Modell} %picture caption
		\label{fig:Funktion Dojo}
	\end{center}
\end{figure}

Das Herzstück des Dōjōs ist ein zentraler NRF52-Mikrocontroller von Nordic Semiconductor mit integriertem und low-Energy fähigem Bluetooth-Stack. Die verwendeten Daten werden auf einer SD-Karte gespeichert. Bei Bedarf werden die Audiodateien an den Verstärker weitergegeben und schliesslich über den Knochenschallaktor wiedergegeben. Die Energieversorgung wird mit einer Li-Ion-Batterie sichergestellt, welche über ein induktives Ladesystem versorgt wird. Abbildung \ref{fig:Teilsysteme} zeigt das technische Lösungskonzept im Überblick.

\begin{figure}[H]
	\begin{center}
		\includegraphics[width=110mm]{data/Loesungskonzept_Teilsysteme.png}
		\caption{Teilsysteme des Dōjōs} %picture caption
		\label{fig:Teilsysteme}
	\end{center}
\end{figure}

Die Funktionen des Dōjōs werden nachfolgend in zwei Bereiche unterteilt. Zuerst wird die Funktionalität aus Sicht des Users beschrieben. Anschliessend werden die relevanten Funktionen aus Sicht der Betreiber erläutert.

Der \textbf{User} geht mit dem Dōjō durch das Museum. Sobald die Bluetooth Beacons genügend nahe sind, erhält der User durch eine aufleuchtende LED ein Signal. Jetzt kann er entscheiden, ob er sich das zugehörige Audio-File anhören will. Trifft dies zu, wird die Audiodatei durch das Betätigen des Play-Buttons abgespielt. Die Lautstärke kann über die Volume-Buttons justiert werden. Falls das Ausstellungsstück dem User gefallen hat, kann die \glqq Like\grqq-Taste gedrückt werden. Dadurch wird das Ausstellungsstück auf einer Liste in der internen SD-Karte gespeichert. Am Ende des Museumsbesuches kann diese Liste ausgewertet werden, wobei die Verwendung dieser Daten nicht Bestandteil der Projektarbeit ist und aus diesem Grund nicht weiterverfolgt wird.

Der \textbf{Betreiber} hat die Aufgabe den Dōjō zu konfigurieren. Dies erfolgt über die dafür vorgesehene SD-Karte, welche mit dem Computer beschrieben wird. Anschliessend kann die Speicherkarte in den Dōjō eingeführt werden. Die Energieversorgung erfolgt über eine induktive Ladestation. Die nächsten beiden Funktionen sind Wunschziele, die vor allem mit Rücksicht auf die Laufzeit realisiert werden. Den Bluetooth-Receiver könnte man kurzzeitig auf ein Bluetooth Beacon umschalten. Der Betreiber müsste nur noch einen Receiver pro Raum installieren. Damit könnte man die gewünschte Lokalisierung der Besucher umsetzen. Das zweite wäre die Möglichkeit per Bluetooth einzelne Audiofiles auf den Dōjō zu übertragen, um im Falle einer Änderung der Ausstellung, die Liste anzupassen und damit aktuell halten zu können.

\subsection{Anwendung}\label{sec:ladeablauf}
Nachdem die Unterteilung zwischen Betreiber und User gemacht wurde, kann nun die Betriebsebene verdeutlicht werden. Um einen lückenlosen Betrieb zu gewährleisten, ist ein Konzept für den Ablauf von Vorteil. Nachfolgend werden die Hauptschritte erklärt. Abbildung \ref{fig:Anwendungsablauf Dojo} dient zur Übersicht und zeigt einen möglichen Anwendungsablauf.

\begin{figure}[H]
	\begin{center}
		\includegraphics[width=120mm]{data/Ladezyklus.png}
		\caption[Anwendungsablauf des Dōjōs]{Anwendungsablauf im Museum} %picture caption
		\label{fig:Anwendungsablauf Dojo}
	\end{center}
\end{figure}

\textbf{Schritt 1} beinhaltet die Ausgabe des Dōjōs am Empfang. Es werden jeweils die Geräte mit dem höchsten Ladestatus abgegeben. Ein lückenloser Betrieb wird erreicht, wenn die Stückzahl der Audio-Guides in etwa der Anzahl Besucher pro Tag entspricht. Die Zutrittsberechtigung wird gemäss dem Wunsch des Besuchers festgelegt. Anschliessend wird die Sprache durch den Besucher selbst gewählt. Hierbei stehen ihm vier Bluetooth-Beacons zur Verfügung, zu welchen er sein Dōjō hinhalten kann. Die gewünschte Sprache ist hierbei durch die Landesflagge gekennzeichnet. Abbildung \ref{fig:SprachauswahlBeacon} zeigt das Prinzip. Damit eine der vier Sprachen auf dem Dōjō aktiviert wird, muss das Gerät an den richtigen Beacon gehalten werden und gleichzeitig die Play-Taste gedrückt werden. Wurde die Sprache erfolgreich ausgewählt, wird ein kurzes Audio-Sample zur Überprüfung der gewählten Sprache abgespielt. Sobald die gewünschte Sprache geladen und getestet wurde, kann der Museumsbesuch gestartet werden.

In \textbf{Schritt 2} befindet sich der Besucher auf dem Rundgang mit dem Dōjō. Dabei hat er die Möglichkeit, während dem Rundgang Bilder zu \glqq liken\grqq und sich die zugehörigen Audio-Files der Kunstobjekte anzuhören.

Die Abgabe der Gerätes erfolgt in \textbf{Schritt 3}. Hier hat der Besucher die Möglichkeit die Kunstobjekte, welche mit einem {\glqq Like\grqq} versehen wurden, als Broschüre oder per Mail zu erhalten. Der entgegengenommene Dōjō kann nun für den nächsten Besucher gereinigt werden.

In \textbf{Schritt 4} werden alle benötigten Informationen extrahiert. Danach wird das Gerät mit der induktiven Ladestation geladen. Für weitere Informationen zur Ladestation wird auf das Kapitel \ref{sec:energieuebertragung} verwiesen. Nun kann der Zyklus wieder von vorne beginnen.
\begin{figure}[H]
	\begin{center}
		\includegraphics[width=140mm]{data/BeaconSpracherkennung.png}
		\caption[Sprachauswahl mittels Bluetooth-Beacon]{Sprachauswahl mittels Bluetooth-Beacon} %picture caption
		\label{fig:SprachauswahlBeacon}
	\end{center}
\end{figure}
\newpage
\section{Hardware}\label{sec:hardware}


\subsection{Energieübertragung}\label{sec:energieuebertragung}


\subsection{Li-Ion-Batterie}\label{sec:energiespeicher}

Die gesamte Energiespeicherung erfolgt durch einen Lithium-Ionen-Akkumulator des Typs Emmerich LI14500. Dieser weist eine Kapazität von $800mAh$ bei einer Nominalspannung von $3.7V$ auf. Ausserdem weist der Akkumulator integrierte Schutzeinrichtungen auf, welche später im Abschnitt \nameref{sec:schutzeinrichtung} (Kapitel \ref{sec:energiespeicher}) weiter erläutert werden. Um eine Abschätzung über die Betriebszeit des Dōjōs zu erhalten, sind Faktoren wie maximaler Verbrauch, Nominalspannung und Kapazität notwendig. Die maximale Leistung des Dōjōs lässt sich durch die Leistung des Knochenschallgebers und die des Microcontrollers beschreiben. Alle anderen Komponenten können durch ihren geringen Betriebsstrom durch einen Sichheitsfaktor von $0.1W$ dazu gerechnet werden. Die Leistung des Verstärkers welcher den Knochenschallgeber speist, weist eine maximale RMS Leistung von $471.9mW$ auf. Die Rechnung erfolgt mit $80\%$ des RMS Wertes, da der Knochenschallgeber nicht rund um die Uhr angesprochen wird. Die Microcontrollerleistung ergab durch tests im Labor eine Leistung von 27.36mW. Nachfolgende Berechnung \ref{eq:MaxLeistung} gibt einen Einblick in die Gesamtleistung.

\begin{equation}
\centering
P_{max}=\left(0.8\cdot P_{Kn}\right)+P_{MC}+P_{zus}=(0.8\cdot 0.472W)+(0.02736W)+0.1W=0.506W
\label{eq:MaxLeistung}
\end{equation}

Die Gesamtleistung beträgt somit rund 0.506W. Um die daraus folgende minimale Zeit t zu berechnen, gilt nachfolgende Berechnung \ref{eq:Betriebszeit}.

\begin{equation}
\centering
t_{max}=\frac{W\cdot U}{P_{tot}}=\frac{800mAh \cdot 3.7V}{506mW}=5.85h\approx 5h \thickspace 51min
\label{eq:Betriebszeit}
\end{equation}

Bei ständigem Gebrauch kann somit eine minimale Betriebszeit von knapp 6 Stunden erreicht werden. Hierbei gilt es zu erwähnen, dass durch einen geschickten Ladeprozess (gemäss Kapitel \ref{sec:anwendung}) ein lückenloser Betrieb garantiert werden kann.


\subsubsection*{Schutzeinrichtungen}\label{sec:schutzeinrichtung}
Um den verwendeten Akkumulator zu schützen, sind diverse Schutzeinrichtung notwendig. Zum einen muss der Ladevorgang überwacht werden, so dass der maximale Ladestrom wie auch die Ladespannung nicht überschritten werden. Für die Laderegelung wurde ein Lade-IC von Microchip des Typs MCP73831 verwendet. Dieser übernimmt die gesamte Spannungs- und Stromregelung beim Ladeprozess und kann zu dem während dem Ladevorgang eine LED zur Ladesignalisation ansteuern. Der Ladeprozess für den oben erwähnten Li-Ion Akku ist in untenstehender Abbildung  \ref{fig:Ladekurve Li-Ion Akku} ersichtlich. Hierbei wurde der Akku im Schnelllademodus mit einem maximalen Strom von 400mA geladen. Dieser Strom ergibt sich aus dem Datenblatt der Batterie, wobei sowohl der Entladestrom, als auch der Ladestrom 0.5C beträgt. Das C entspricht der Kapazität der Batterie, wodurch sich der Strom Imax gemäss der nachfolgenden Formel \ref{eq:Ladestrom} berechnen lässt.

\begin{equation}
\centering
I_{charge}={\frac{0.5}{h} \cdot C}={\frac{0.5}{h} \cdot 800mAh}=0.4A\thickspace \widehat {=} \thickspace 400mA
\label{eq:Ladestrom}
\end{equation}

Betrachtet man die Abbildung \ref{fig:Ladekurve Li-Ion Akku}, so wird ersichtlich, dass die Spannung rund $2.5h$ geregelt wird bis 4.2V Grenze erreicht wird. Sobald der Spannungswert $4.2V$ erreicht hat, beginnt der Lade-IC mit der Stromregelung. Für diesen Prozess wurden beim Versuch noch einmal rund 30 Minuten benötigt, wodurch die letzten rund $20\%$ der Batteriekapazität geladen werden konnten.

\begin{figure}[H]
	\begin{center}
		\includegraphics[width=120mm]{data/LadekurveLiIon.png}
		\caption[Blockschaltbild Energiespeicherung]{Blockschaltbild Energiespeicherung} %picture caption
		\label{fig:Ladekurve Li-Ion Akku}
	\end{center}
\end{figure}


Für einen weiteren Schutz, hat die Emmerich LI14500 eine integrierte Schutzbeschaltung namens PCM (Protection Circuit Module). Dieser Schutz garantiert einerseits einen Überladeschutz von $4.25V\pm 0.025V$, aber auch einen Tiefentladungsschutz von $2.5V\pm 0.063V$. Weiter ist der Akku gegen Überströme ab einer Höhe von 4.8A geschützt und weist zudem einen Schutzschaltungswiderstand von $\leq 75mW$ auf. Das Testing des integrierten Schutzschaltung findet in der Hardware Validierung \ref{sec:testkonzeptHardware} statt.



\subsection{Ladestation} \label{sec:ladestation}

Nachdem die gesamte induktive Ladeschaltung und Energiespeicherung beschrieben wurde, folgt die Beschreibung der Ladestation. Hierfür wurde ein Prototyp erstellt, welche zum einen die gesamte primäre induktive Ladeschaltung beinhaltet, zum anderen eine Aussparung, welche als Ladeeinrichtung für den Dōjō dient, vorweist. Für den Prototypen wurde eine .stl Datei erstellt welche mit einem 3D-Drucker gedruckt wurde. Wichtig ist hierbei zu erwähnen, dass es sich bei nachfolgende Abbildungen \ref{fig:Prototyp Front} bis \ref{fig:Prototyp Down} nur um Prototypen handelt und es bei einer Weiterentwicklung noch Anpassungen geben kann. Da die Ladestation für Versuchszwecke betreffend der induktiven Ladeschaltung bereits erstellt wurde, ist das Design so gewählt, dass nur ein Dōjō geladen werden kann. Dies könnte in einem weiteren Schritt auf mehrere Ladeaussparungen erweitert werden, wobei mehrere Dōjōs gleichzeitig pro Ladestation geladen werden können. 

\begin{figure}[H]
	\begin{center}
		\includegraphics[width=80mm]{data/DojoLadestation01.png}
		\caption[Prototyp Ladestation Frontansicht]{Frontansicht - Ladestation Dōjō} %picture caption
		\label{fig:Prototyp Front}
	\end{center}
\end{figure}

Die oben gezeigte Abbildung gibt einen Einblick in das Design von vorne. Augenfällig ist die Öffnung für den Dōjō selbst, welche zum einen als Standhalterung und zum anderen als korrekte Positionierung für die induktive Ladeschaltung dient. Die richtige Positionierung ist hierbei eines der wichtigsten Kriterien für einen optimalen Ladezyklus, da die Tranceiver- und Receiverspule direkt übereinanderliegend den besten Wirkungsgrad erzielen. Weiter ist in der Abbildung der Dōjō Schriftzug ersichtlich, welcher bis in die dahinter liegende Kammer führt. Die Verwendung dieser Durchführung wird nachfolgend unter der Abbildung \ref{fig:Prototyp Down} weiter erklärt. 


\begin{figure}[H]
	\begin{center}
		\includegraphics[width=80mm]{data/DojoLadestation02.png}
		\caption[Prototyp Ladestation Draufsicht]{Draufsicht - Ladestation Dōjō} %picture caption
		\label{fig:Prototyp Top}
	\end{center}
\end{figure}

Die obige Abbildung \ref{fig:Prototyp Top} zeigt den Prototypen von oben. In der Aussparung für den Dōjō ist ein Kanal für die Verkabelung der Primärspule ersichtlich. Diese Öffnung führt zur Kammer welche den Primär-Ladekreises beinhaltet. Die Abmessung (Länge x Breite x Höhe) des Prototypen ist (80 x 70.7 x 30)mm. Einen Einblick in die Kammer für die Elektronik des Primär-Ladekreises gibt nachfolgende Abbildung \ref{fig:Prototyp Down}, welche den Prototypen von unten zeigt.

\begin{figure}[H]
	\begin{center}
		\includegraphics[width=80mm]{data/DojoLadestation03.png}
		\caption[Prototyp Ladestation Ansicht von Unten]{Ansicht von Unten - Ladestation Dōjō} %picture caption
		\label{fig:Prototyp Down}
	\end{center}
\end{figure}

Die grosse Kammer ist wie bereits oben beschrieben für den Primärkreis der induktiven Ladeschaltung vorgesehen. Für die Validierung wurde eine Lochrasterplatine mit allen nötigen Komponenten gefertigt, welche genau in diese Aussparung passt. Weiter sind runde Löcher (5mm $\o$) ersichtlich, welche in die Kammer des Schriftzuges führen. Diese Durchführungen sind für LEDs vorgesehen, welche den Dōjō Schriftzug bei angeschlossener Versorgungsspannung zum Leuchten bringt.
\subsection{Mikrocontroller} \label{sec:microcontrollerHardware}

Als Microcontroller wurde ein nRF52832 der Firma Nordic Semiconductors verwendet ( ersichtlich in Abbildung \ref{fig:nRF52832}). Seine hohe Performance ermöglicht es ein System aufzubauen, welches den Microcontroller als zentrale Schnittstelle beinhaltet. Der Microcontroller bildet wie in Kapitel \ref{sec:gesamtkonzept} gemäss Abbildung \ref{fig:Teilsysteme} bereits beschrieben das Herzstück des Dōjōs. 

\begin{figure}[H]
	\begin{center}
		\includegraphics[width=60mm]{data/nRF52832.png}
		\caption[nRF52832 Microcontroller]{nRF52832 Microcontroller \cite{nRF52832}} %picture caption
		\label{fig:nRF52832}
	\end{center}
\end{figure}

der nRF52832 weist eine Betriebsversorgungsspannung zwischen 1.7V und 3.6V mit einem Versorgungsstrom von 5.4mA auf. Der Speicher des nRF52832 ist durch 512kB flash/64kB RAM Speicher gegeben. 
Für die ersten Tests der Software wurde ein Development Kit (Abbildung \ref{fig:nRF52832-DK}) mit intergriertem nRF52832 verwendet. Die Beschreibung der Software ist in Kapitel \ref{sec:software} ersichtlich.
\begin{figure}[H]
	\begin{center}
		\includegraphics[width=100mm]{data/NRF52-DK.png}
		\caption[nRF52 Development Kit]{nRF52 Development Kit \cite{nRF52-DK}} %picture caption
		\label{fig:nRF52832-DK}
	\end{center}
\end{figure}
 Die Schlüsselmerkmale dieses Boards sind zum einen wie bereits beschrieben der integrierte Microcontroller, zum anderen aber auch eine integrierte Bluetooth Antenne. Das Kit unterstützt proprietäre Bluetooth Smart, ANT und 2.4GHz Applikationen. Zudem ermöglicht es den Einsatz von Drittanbieter Shields. Dies ermöglicht das Beschreiben und Lesen der SD-Karte zu testen.

\subsection{Verstärkerstufe} \label{sec:verstaerkerstufe}
Nachdem die Steuerungseinheit gezeigt wurde, kann nun die Ausgangsstufe erläutert werden. Mit einer Verstärkerstufe lassen sich auf einfache Art und Weise Signale jeglicher Form verstärken. Sie eignen sich bestens, um den Ausgang eines Mikrocontrollers entsprechend aufzubereiten, da die Ausgangsseite meist sehr niedrige Ströme aufweist. Dadurch kann dem Knochenschallaktor genügend Energie zur Verfügung gestellt werden. Prinzipiell gibt es zwei Arten von Verstärkern. Entweder erfolgt die Umsetzung digital oder analog. Beide erfüllen die gleiche Aufgabe, weisen jedoch bezüglich Wirkungsgrad einen deutlichen Unterschied auf. Die digitale Variante weist ungefähr einen Wirkungsgrad von 90$\%$ auf \cite{BoneConductorAdafruit}, während die analoge Variante einen maximalen Wirkungsgrad im Bereich der Leistungsanpassung erzielt \cite{Niklaus_Skript}. Aus diesem Grund wird ein digitaler Verstärker (Class-D-Verstärker) in der Anwendung implementiert. Die Wahl fiel auf den Stereo-Amplifier MAX 98306. Der Verstärker hat einen Stromverbrauch von $143mA$ und eine Speisespannung von $3.3V$. Somit hat er einen Leistungsverbrauch von $471.9 mW$. Im Standby benötigt er lediglich $2 mA$ und somit $6.6 mW$.
\subsection{Knochenschallaktor} \label{sec:knochenschallaktor}
\newpage
\section{Software}\label{sec:software}
Wichtig für dieses Kapitel ist, dass der beschriebene Zustand der Software dem Soll-Zustand entspricht. Die Abweichungen sind im Kapitel Validierung genauer beschrieben.

Es wird ein NRF52832 verwendet von Nordic Semiconductor. Dadurch liegt die Verwendung des Software Development Kit \cite{nordic_sdk} nahe. Dies ist eine Sammlung von Beispielen, Librarys und vorcompilierten Codes. Nachfolgend wird das Software Development Kit nur noch SDK genannt. Es wurde nRF5 SDK v12.3.0 verwendet. Wichtig ist die SDK für ihren integrierten Bluetooth-Stack, der verwendet wird. Dieser ist im sogenannten Softdevice enthalten. Um ihn nutzen zu können, verwenden wir den S132. Dieser und die nötige Initialisierungen des Bluetooth-Stacks waren in dem Beispielprojekt Uartc in der Central-Rolle vorhanden. Dadurch wurde das ganze Projekt auf diesem Beispiel aufgebaut. Der Softdevice und die SDK legen einige abstraktions Layer auf die Hardware. Diese sind in Abbildung \ref{fig:Software_Layers} visualisiert. Die wichtigsten Module des SDK werden im Kapitel \ref{sec:nordicsdk} erklärt, falls weitere Informationen gewünscht sind wird auf die offizielle Dokumentation verwiesen \cite{nordic_info}.

\begin{figure}[htbp!!!!]
	\centering
	\includegraphics[width=0.7\textwidth]{Data/Software_Layers.PNG}
	\caption[Software:Layers]{Aufbau der Software auf der Hardware}
	\label{fig:Software_Layers}
\end{figure} 

Der eigentliche Programmaufbau ist eine Statmachine. Diese ist in einem separaten Kapitel erläutert. Das Programm ist so aufgebaut, dass alle Events möglichst kurz gehalten wurden. Jedoch hat dies relativ viele Flags zur folge. Anschliessend werden die die Events entsprechend ihrer Priorität verarbeitet. Die Prioritäten ergeben sich aus der Else IF im Wait State, genauers im Kapitel \ref{sec:stateMachine}. Dadurch entsteht ein pollendes Programm mit Prioritäten, was in Abbildung \ref{fig:Software_approach} dargestellt ist.

\begin{figure}[htbp!!!!]
	\centering
	\includegraphics[width=0.7\textwidth]{Data/Software_Pollend.pdf}
	\caption[Software:Pollend]{Konzept der pollenden Software}
	\label{fig:Software_approach}
\end{figure}

Das Programm wurde in verschiedene Module unterteilt, um die Leserlichkeit zu verbessern. Die Unterteilung wurde entsprechend der Funktion gemacht. Im Hauptmodul (Main) ist nur die Statmachine und die Interrupthandler vorhanden. Folglich wurde ein Modul für die BLE-Funktionalitäten geschrieben und eines für die SD-Karte. Ein weiters existiert für die Batterie-Funktionalitäten. Dieses ist jedoch nicht implementiert und enthält nur Dummy-Funktionen.







\subsection{State-Machine}\label{sec:stateMachine}

Nachdem mit dem Einleitungskapitel ein kurzer Überblick geschaffen wurde, kann nun der Hauptteil der Software genauer betrachtet werden. Der gesamte Ablauf basiert auf einer klassischen State-Machine, die aufgrund von unterschiedlichen Parametern in die entsprechenden nächsten States springt. Das hat den Vorteil, dass sich das Programm stets in einem definierten Zustand befindet und mittels entsprechenden Parametern jeweils den nächsten Arbeitsschritt vordefiniert. Die Abbildung \ref{fig:completeStateMachine} zeigt das Gesamtkonzept der State-Machine. Anschliessend werden die einzelnen States genauer definiert und beschrieben.

\begin{figure}[htbp]
	\centering
	\includegraphics[width=1.15\textwidth]{Data/StateMachineFinal.pdf}
	\caption[Statemachine-Diagramm]{Statemachine im Überblick mit den einzelnen States und den Parametern}
	\label{fig:completeStateMachine}
\end{figure} 

\subsubsection*{State: Lookup}
In diesem State verschafft sich das Programm über die entsprechende Initialisierung Zugriff auf die SD-Karte der Anwendung. Falls der Mikrocontroller nicht auf die SD-Karte zugreifen kann, wird eine Fehlermeldung ausgegeben und die Funktion wird beendet. Anderenfalls wird dem Mikrocontroller signalisiert, dass der Zugriff geglückt ist und die eigentliche Funktion wird gestartet. Dazu werden die beiden Minor- und Majorzahlen in ein hexadezimales Zahlensystem gewandelt, welche dann als Vergleichskriterium verwendet werden. Falls die Nummer gefunden wird, kann das entsprechend zugehörige Audio-File über den Mikrocontroller ausgegeben werden. Verglichen wird jeweils zeilenweise, weshalb auch ein Fehlerhandling eingebaut wurde. Damit wird erkannt, ob sich das Textfile am Ende befindet. Somit lässt sich dann die Suche wiederholen, oder einen Fehler ausgeben. Die nachfolgende Abbildung \ref{fig:lookupState} zeigt den detaillierten Funktionsablauf im Lookup-State.

\begin{figure}[htbp!!!!]
	\centering
	\includegraphics[width=0.57\textwidth]{Data/lookup_picture}
	\caption[Statemachine: lookup]{Funktionsablauf im lookup-State}
	\label{fig:lookupState}
\end{figure} 

\subsubsection*{State: Special Beacon}
Special Beacon dient hauptsächlich zur Unterscheidung der verschiedenen Beacons für die Sprache, das Drucken und weitere Features die in einem weiteren Ansatz implementiert werden können. Aus diesem Grund wurde dieser State auch relativ einfach gehalten. Zuerst wird verglichen, ob es sich dabei um die Sprachkonfiguration handelt. Ist dies zutreffend, so springt das Programm in den Change Language State. Anderenfalls wird überprüft, ob gerade der Like-Button gedrückt wird und entsprechend in den Print \glqq Likes \grqq State gewechselt wird. Ist keine der beiden Zustäde zutreffend, springt das Programm in den Wait State. Die nachfolgende Abbildung \ref{fig:specialBeaconState} zeigt den Ablauf der Funktion.

\begin{figure}[htbp!!!!]
	\centering
	\includegraphics[width=0.9\textwidth]{Data/SpecialBeacon_picture.pdf}
	\caption[Statemachine: Special Beacon]{Funktionsablauf im Special Beacon State}
	\label{fig:specialBeaconState}
\end{figure} 
\newpage
\subsubsection*{State: Change Language}

Dieser State ist für die Sprachauswahl verantwortlich. Aufgrund des Zahlenwertes in Minor [1] wird zwischen den Landessprachen der Schweiz ausgewählt. Dabei wird die Vergleichstabelle in Form einer Datei kopiert und mit einem Kürzel entsprechend der Sprache versehen. Danach springt das Programm wieder in den lookup State.

\begin{figure}[htbp!!!!]
	\centering
	\includegraphics[width=0.7\textwidth]{Data/ChangeLanguage_picture.pdf}
	\caption[Statemachine: Change Language]{Funktionsablauf im Change Language State}
	\label{fig:changeLanguageState}
\end{figure} 

\subsubsection*{State: Print \glqq Likes \grqq}

Print \glqq Likes \grqq wurde nicht implementiert und das Hauptprogramm sprint in den Wait State. Das ist eine optionale Möglichkeit, die in einem weiteren Entwicklungskonzept bearbeitet werden kann. Die Software ermöglicht es aber diese Funktion noch einzubetten. Dabei könnte die Broschüre mit den interessanten Objekten direkt gedruckt werden und bereits für den Besucher am Ausgang des Museums bereit liegen.

\subsubsection*{State: Wait}

[Warte auf Infos Loosli/Elias, noch unklar]

\subsubsection*{State: New Beacon}

In diesem State wird der Ringbuffer ausgelesen. Dadurch lässt sich das Beacon mit dem stärksten Signal identifizieren. Ist das aktuelle Beacon immer noch das gleiche wie das alte Beacon, springt das Programm in den Wait State. Wenn es sich nicht um das gleiche Beacon handelt, wird über den lookup State überprpüft, ob eine entsprechende Übereinstimmung vorhanden ist.

\begin{figure}[htbp!!!!]
	\centering
	\includegraphics[width=0.5\textwidth]{Data/NewBeacon_picture.pdf}
	\caption[Statemachine: New Beacon]{Funktionsablauf im New Beacon State}
	\label{fig:newBeaconState}
\end{figure} 

\subsubsection*{State: Like}

Wird in diesem State über einen Button ein Like-Ereignis ausgelöst, springt das Programm in den Signal to User State. Findet kein Ereignis statt, dann wird in den Wait State gesprungen.

\begin{figure}[htbp!!!!]
	\centering
	\includegraphics[width=0.5\textwidth]{Data/Like_picture.pdf}
	\caption[Statemachine: Like]{Funktionsablauf im Like State}
	\label{fig:likeState}
\end{figure} 

\subsubsection*{State: Signal to User}

Dieser State hat die Aufgabe, dem Benutzer mitzuteilen, dass ein neues Audio-File verfügbar ist. Dabei blinkt eine dafür vorgesehene LED. Anschliessend springt die Software in den Wait State.

\begin{figure}[htbp!!!!]
	\centering
	\includegraphics[width=0.2\textwidth]{Data/SignalToUser_picture.pdf}
	\caption[Statemachine: Signal to User]{Funktionsablauf im Signal to User State}
	\label{fig:signalToUserState}
\end{figure} 

\subsubsection*{State: ADC Battery}

[Mehr Informationen benötigt, gemäss Code passiert da gar nichts]

\subsubsection*{State: Shutdown}

[Mehr Informationen benötigt, gemäss Code passiert da gar nichts]

\subsubsection*{State: Merken Liste löschen}

Dieser State löscht die Likes, um für den nächsten User bereit zu sein. Anschliessend folgt der Charge State.

\begin{figure}[htbp!!!!]
	\centering
	\includegraphics[width=0.2\textwidth]{Data/MerkeListeLoeschen_picture.pdf}
	\caption[Statemachine: Merke Liste löschen]{Funktionsablauf im Merke Liste löschen State}
	\label{fig:merkeListeLoeschenState}
\end{figure} 

\subsubsection*{State: Charge}

[Mehr Informationen benötigt, gemäss Code passiert da gar nichts]

\subsubsection*{State: Pause}

Dieser State pausiert das Sound-File und wartet auf ein weiteres Play-Event, um die Audiodatei wieder abzuspielen.

[Infos von Elias benötigt um das Diagramm zu erstellen]

\subsubsection*{State: Init Buffer}

[Infos von Elias benötigt um das Diagramm zu erstellen]

\subsubsection*{State: Bufferreload}

[Infos von Elias/Loosli benötigt um das Diagramm zu erstellen]
\subsection{Nordic Software Development Kit}\label{sec:nordicsdk}

Beschreibt die Aufgabe des MCs mittels Ablaufdiagramm, sowie welche Funktionen dazugehören und deren Wirkung.
Funktionen, Prinzip des Kits, Example des ursprünglichen Aufbaus
\subsection{Bluetooth}\label{sec:bluetooth}

\subsubsection{Bluetooth Grundlagen}
Standard-Bluetooth-Geräte senden in einem lizenzfreien Band zwischen 2.402 und 2.480 GHz. Dabei können Störungen durch diverse andere Geräte auftreten, welche im selben Frequenzband arbeiten. Um eine Robustheit gegenüber Störungen zu erhalten, wird ein Frequenzsprungverfahren eingesetzt. Bei diesem Verfahren wird das Frequenzband in 79 Kanäle eingeteilt und bis zu 1600-mal in der Sekunde gewechselt \cite{5_Teildokument_BT}.

Derzeitiger Bluetooth-Standard ist Bluetooth 5. Bluetooth 5 besitzt im Gegensatz zu seinen Vorgängern eine höhere Reichweite (bis zu 100 m statt 25 m) und eine schnellere Datenrate (bis zu 2 Mbit/s statt 1 Mbit/s). Ausserdem enthält dieser Standard ebenso den im Standard 4 bereits eingeführten «Low Energy»-Modus, welcher den schon moderaten Energieverbrauch zusätzlich senkt, was für dieses Projekt benötigt wird \cite{5_Teildokument_BT}. Im Low-Energy-Modus wird für das Frequenzsprungverfahren das Frequenzband nicht in 79- sondern in 40 Kanäle unterteilt. Ausserdem wird auch bei der Datenrate Energie gespart, indem eine Geschwindigkeit von 1- statt 2 Mbit/s erreicht werden kann. Die typische Reichweite im Low-Energy-Modus beträgt 40 m, was die Mindestanforderung von 5 Metern für das Projekt überschreitet und deshalb genügt \cite{6_Teildokument_BT}. Damit über Bluetooth Low Energy (BLE) Verbindungen aufgebaut werden können, benötigt es sogenannte Profile. Bei Bluetooth sind dies das GAP (Generic Access Profile) und das GATT (Generic Attribute Profile). Auf diese wird in den nächsten Kapiteln näher eingegangen.

\subsubsection{Generic Access Profile}
Das GAP kontrolliert Verbindungen und Authentifizierungen. Es beschreibt wie Geräte miteinander kommunizieren. Dazu definiert es zwei verschiedene Rollen für die Geräte. Zum einen die Rolle des zentralen Geräts und zum anderen die Rolle des Peripheriegeräts \cite{7_Teildokument_BT}. Damit eine Verbindung zustande kommt, muss das Peripheriegerät in einem bestimmten Intervall ein Datenpaket das Payload genannt wird, senden. Empfängt ein zentrales Gerät das Payload, kann es für mehr Daten zusätzlich ein Antwortpaket (scan response payload) anfordern, welches daraufhin vom Peripheriegerät gesendet wird \cite{7_Teildokument_BT}. Meistens senden die Peripheriegeräte ihre Authentifizierung, damit eine Verbindung eingegangen und das GATT für mehr Datenaustausch benutzt werden kann. Während diese Verbindung besteht, kann das Peripheriegerät keine weiteren Verbindungen eingehen. Wird jedoch nur eine kleine Datenmenge benutzt, kann diese in das Payload integriert werden, womit alle zentralen Geräte in der Nähe Zugriff auf die Daten haben. Dieses Verfahren nennt man Broadcasting und ist für dieses Projekt von zentraler Bedeutung. Dennoch wird im nächsten Kapitel erläutert, wozu das GATT nützlich ist \cite{7_Teildokument_BT}.

\subsubsection{Generic Attribute Profile}
Das GATT definiert die Art und Weise, wie Peripherie und zentrales Gerät miteinander Daten austauschen. Dies macht es mit Hilfe des Attribute Protocol (ATT) \cite{8_Teildokument_BT}. Das ATT speichert Services, Characteristics und dazugehörende Daten. Services sind logische Sammlungen von zusammengehörenden Characteristics, wobei Characteristics als Datenpunkte angesehen werden können. Demzufolge besteht das ATT aus einer in Services geordneten Datensammlung \cite{8_Teildokument_BT}.

Peripheriegeräte speichern das GATT und dienen somit als GATT-Server. Zentrale Geräte verbinden sich mit Hilfe des GAP mit dem GATT-Server und stellen eine Anfrage. Das zentrale Gerät wird somit zum GATT-Client. In einem Intervall werden vom Client gesendete Anfragen vom Server mit Datenpakete beantwortet. So können grössere Datenpakete ausgetauscht werden, jedoch kein Broadcasting stattfinden \cite{8_Teildokument_BT}. Das genannte zentrale Gerät bzw. GATT-Client ist der Dojo. Die genannten Peripheriegeräte sind BLE-Beacons, welche in der Nähe der Kunstobjekte angebracht sind. Im nächsten Kapitel wird näher auf die BLE-Beacons eingegangen \cite{8_Teildokument_BT}.

\subsubsection{BLE-Beacons}
BLE-Beacons (Bluetooth Low Energy Beacons) sind kleine Peripheriegeräte, welche dazu da sind kleine Informationsmengen zu übertragen. Wird z.B. ein Temperaturverlauf über ein Jahr hinweg gemessen, so kommen BLE-Beacons zum Einsatz, da diese mit einer Knopfzellenbatterie über Jahre hinweg in Betrieb sein können und kleine Informationsmengen übertragen können \cite{9_Teildokument_BT}.

Generell können BLE-Beacons vier Rollen einnehmen. Diese vier Rollen kann man unterteilen in zwei, welche Verbindungen eingehen können, und zwei weitere, welche nicht dazu in der Lage sind. Die Rollen, welche Verbindungen eingehen können, sind zum einen die eines Peripheriegeräts und zum anderen die eines zentralen Geräts. Die, welche keine Verbindungen eingehen können, sind die eines Senders und die eines Beobachters. Als Peripheriegerät fungiert das BLE-Beacon als Slave und wartet somit auf Input des zentralen Geräts (des Masters). Als zentrales Gerät fungiert das BLE-Beacon als Master und kann Verbindungen mit einem oder mehreren anderen Geräten eingehen. Als Sender (Broadcaster) kann das BLE-Beacon zwar keine Verbindngen eingehen, aber z.B. oben erwähnte gemessene Temperatur oder einen vordefinierten Wert senden. Wird das BLE-Beacon als Beobachter (Observer) benutzt, so kann dieser z.B. die von einem anderen BLE-Beacon gesendeten Werte empfangen und an einem angeschlossenen Display anzeigen, jedoch auch keine Verbindungen eingehen \cite{9_Teildokument_BT}. Die übermittelten Signale der BLE-Beacons werden gemäss Abbildung \ref{fig:PacketPayload_Header} formatiert \cite{9_Teildokument_BT}.

\begin{figure}[htbp!!!!]
	\begin{center}
		\includegraphics[width=\textwidth]{data/PacketPayload_Header.png}
		\caption[PacketPayload Header]{PacketPayload Header} %picture caption
		\label{fig:PacketPayload_Header}
	\end{center}
\end{figure}

Die Preamble wird zur Synchronisierung und Zeitschätzung benötigt und ist für BLE-Beacons, welche in der Rolle eines Broadcasters sind, immer 0xAA. Auch die Access Address ist für Broadcaster immer gleich, nämlich 0x8E89BED6. Das Packet Payload beinhaltet Header und Payload. PDU bestimmt den Sendekanaltyp des sendenden Beacons, gemäss folgender Tabelle \ref{tab:PDU} \cite{9_Teildokument_BT}:

\begin{table}[htbp!!!]
\begin{tabular}{|c|c|c|}
\hline 
\rule[-1ex]{0pt}{2.5ex} PDU Type & Packet Name & Description \\ 
\hline 
\rule[-1ex]{0pt}{2.5ex} 0000 & ADV{\_}IND & Connectable undirected advertising event \\ 
\hline 
\rule[-1ex]{0pt}{2.5ex} 0010 & ADV{\_}NONCONN{\_}IND & Non-connectable undirected advertising event \\ 
\hline 
\rule[-1ex]{0pt}{2.5ex} 0110 & ADV{\_}SCAN{\_}IND & Scannable undirected advertising event \\ 
\hline 
\end{tabular} 
\caption[PDU Type]{PDU Type}
\label{tab:PDU}
\end{table}

RFU (Reserved for Future Use) steht, wie der Name schon sagt, als Reserve für zukünftige weitere Implementationen und wird deshalb derzeit nicht gebraucht. TxAdd definiert, ob die Sendeadresse des Beacons Öffentlich (TxAdd = 0) oder zufällig (TxAdd = 1) ist. Das RxAdd tangiert Beacons nicht und wird deshalb nicht weiter erwähnt. Das CRC (Cyclic Redundancy Check) dient der Erkennung von Fehlübertragungen. Das Payload ist gemäss Abbildung \ref{fig:PacketPayload_Payload} aufgebaut \cite{9_Teildokument_BT}.

\begin{figure}[htbp!!!!]
	\begin{center}
		\includegraphics[width=0.8\textwidth]{data/PacketPayload_Payload.png}
		\caption[PacketPayload Payload]{PacketPayload Payload} %picture caption
		\label{fig:PacketPayload_Payload}
	\end{center}
\end{figure}

Es ist zu sehen, dass es mit der Sendeadresse und den Sendedaten gefüllt ist. Die Sendeadresse kann öffentlich oder zufällig sein, wobei eine öffentliche Sendeadresse eine OUI (Organizationally Unique Identifier) nutzt, welche von der IEEE Registration Authority vergeben wird. Die Sendedaten können gemäss der  Tabelle \ref{tab:AD_Data_Type} formatiert werden \cite{9_Teildokument_BT}.

\begin{table}[htbp!!!]
\begin{tabular}{|c|c|c|}
\hline 
\rule[-1ex]{0pt}{2.5ex} AD Data Type & Data Type Value & Description \\ 
\hline 
\rule[-1ex]{0pt}{2.5ex} Flags & 0x01 & Device discovery capabilities \\ 
\hline 
\rule[-1ex]{0pt}{2.5ex} Service UUID & 0x02 - 0x07 & Device GATT services \\ 
\hline 
\rule[-1ex]{0pt}{2.5ex} Local Name & 0x08 - 0x09 & Device name \\ 
\hline 
\rule[-1ex]{0pt}{2.5ex} TX Power Level & 0x0A & Device output power \\ 
\hline 
\rule[-1ex]{0pt}{2.5ex} Manufacturer Specific Data & 0xFF & User defined \\ 
\hline 
\end{tabular} 
\caption[AD Data Type]{AD Data Type}
\label{tab:AD_Data_Type}
\end{table}

Die Flags definieren die Fähigkeiten des BLE-Beacons und sind gemäss der Tabelle \ref{tab:Flags} definiert \cite{9_Teildokument_BT}.

\begin{table}[htbp!!!]
\begin{tabular}{|c|c|c|c|}
\hline 
\rule[-1ex]{0pt}{2.5ex} Byte & Bit & Flag/Value & Description \\ 
\hline 
\rule[-1ex]{0pt}{2.5ex} 0 & • & 0x02 & Length of this data \\ 
\hline 
\rule[-1ex]{0pt}{2.5ex} 1 & • & 0x01 & GAP AD Type Flags \\ 
\hline 
\rule[-1ex]{0pt}{2.5ex} 2 & 0 & LE Limited Discoverable Mode & 180 s advertising \\ 
\hline 
\rule[-1ex]{0pt}{2.5ex} • & 1 & LE General Discoverable Mode & Indefinite advertising time \\ 
\hline 
\rule[-1ex]{0pt}{2.5ex} • & 2 & BR/EDR Not Supported & • \\ 
\hline 
\rule[-1ex]{0pt}{2.5ex} • & 3 & Simultaneous LE and BR/EDR (Controller) & • \\ 
\hline 
\rule[-1ex]{0pt}{2.5ex} • & 4 & Simultaneous LE and BR/EDR (Host) & • \\ 
\hline 
\rule[-1ex]{0pt}{2.5ex} • & 5-7 & • & Reserved \\ 
\hline 
\end{tabular} 
\caption[Flags]{Flags}
\label{tab:Flags}
\end{table}

Die Manufacturer Specific Data können gemäss Tabelle \ref{tab:AD_Data_Type} vom Benutzer definiert werden. Hier können Werte gespeichert werden, die das BLE-Beacon senden soll, wie z.B. den RSSI-Wert. Dieser wird im nächsten Kapitel erläutert \cite{9_Teildokument_BT}.

\subsubsection{RSSI}
RSSI steht für Received Signal Strength Indicator und ist ein Indikator für die Empfangsfeldstärke bei drahtloser Kommunikation. Je höher dieser Wert ist, desto stärker ist das empfangene Signal \cite{10_Teildokument_BT}. Die drahtlose Kommunikation erfolgt über das Senden von elektromagnetischen Wellen. Trifft eine solche Elektromagnetische Welle auf eine Antenne, so führt dies zu einer messbaren Selbstinduktion, welche dem RSSI-Wert entspricht. Dieser Wert muss abhängig von der jeweiligen Anwendung interpretiert werden, da diverse Faktoren bezüglich Transmitter, Receiver und Übertragungsmedium den Wert beeinflussen können.

Im Falle des Museums mit gleichwertigen Beacons bedeutet dies, dass der Besucher sich eher in der Nähe des Beacons mit stärkerem RSSI Wert befindet. Dadurch kann das richtige Audiofile abgespielt werden. Zum Unterscheiden der Beacons werden UUID (Universal Unique Identifier), Major und Minor verwendet. UUID ist im Falle des Projekts eine gewählte Identifikationsnummer für das Museum. Major beschreibt eine eindeutig definierte Zimmernummer und Minor ist die definierte Nummer des Beacons im gleichen Raum. Im nächsten Kapitel wird die Minor- und die Majorvergabe beschrieben.

\subsubsection{Major Minor Vergabe}
Um die Major Minor Nummern mit dem Dōjō zu benutzen, wurden sie im Verlaufe des Projektes standardisiert. Die Sprachauswahl funktioniert über solche Nummern. Beacons mit diesen speziellen Nummern lösen auf dem Dōjō entsprechende Funktionen aus. Abbildung \ref{fig:Bluetooth_def_MM} zeigt die Definitionen. Die Nummernräume sind so gestaltet, dass genug Platz für zusätzliche Funktionen vorhanden ist, wie zum Beispiel für weitere Sprachen oder Zugangskontrollen.

\begin{figure}[htbp!!!!]
	\centering
	\includegraphics[width=0.3\textwidth]{Data/Reserviert_picture.png}
	\caption[Software:Definierte MM]{Default Definitionen von speziellen Major Minor Nummern.}
	\label{fig:Bluetooth_def_MM}
\end{figure}

In Abbildung \ref{fig:Bluetooth_MM_Vergabe} ist zu sehen, dass die erste Major Nummer 0x00 sein muss, damit ein spezieller Beacon erkennt wird. Das vereinfacht das Programm, da bei einem neuen Ble-Package nur der erste Eintrag im Array betrachtet werden muss, um die speziellen Beacons zu erkennen. Der Nummernraum für spezielle Beacons umfasst ca 16 Mio. Adressen, was für zusätzliche Funktionen reichen sollte. Sommit bleiben ca 4.2 Mia. Adressen für Kunstwerke übrig.

\begin{figure}[htbp!!!!]
	\centering
	\includegraphics[width=0.9\textwidth]{Data/Speicheradressen_picture.png}
	\caption[Software:MM Vergabe]{Definition der Majo Minor Vergabe}
	\label{fig:Bluetooth_MM_Vergabe}
\end{figure}




\subsection{SD-Karte}\label{sec:sdKarte}


\subsection{Audio, PWM}\label{sec:audioPWM}

Beschreibt, wie das gewünschte File geholt und auf den Knochenschallgeber gegeben wird.
\subsection{Lizenzen}\label{sec:lizenzen}







\newpage
\section{Validierung} \label{sec:validierung}
\subsection{Testkonzept Hardware}\label{sec:testkonzeptHardware}
Damit ein reibungsloser Betrieb möglich ist, müssen die einzelnen Hardware Komponenten auf Herz und Nieren geprüft werden. Nachfolgend werden die Testverfahren genauer beschrieben und die Testergebnisse aufgelistet.

\subsubsection*{Schutzmechanismen Batterie}\label{sec:batterie}
Die Batterie weist einige Schutzmechanismen auf, welche alle getestet werden müssen. Als erstes wurde der Tiefentladungsschutz geprüft. Um dies zu testen wurde ein Winderstand der Dimension 9$\Omega$ angeschlossen, wobei gemäss Berechnung \ref{eq:Entladestrom} ein Entladestrom von rund 400mA resultierte.

\begin{equation}
\centering
I_{discharge}=\frac{U}{R}=\frac{3.7V}{9\Omega }= 411mA
\label{eq:Entladestrom}
\end{equation}

Während dem Entladevorgang wurde stets die Spannung überwacht, wobei die Spannung von 3.7V auf bis 2.5V absank. Nach dem die 2.5V Schwellenspannung unterschritten wurde, brach der integrierte Batterieschutz die Spannungsversorgung ab. Die Widerstände wurden abgehängt und der gesamte Vorgang wurde mit Erfolg wiederholt.
\\
Als nächstes wurde ein Kurzschlusstest durchgeführt, wobei hier der Schwellenstrom gemäss Datenblatt bei 4.8 liegt. Gemäss dem U=R$\cdot$I Gesetz, wurde ein Widerstand der Grösse von 700m$\Omega$ verwendet damit der Grenzwert überschritten wird. Auch bei diesem Versuch, riegelte das PCM den hohen Entladungsstrom ab und schaltete die Versorgungsspannung ab.

\subsubsection*{Ladeschaltung der Batterie}\label{sec:batterie}
Für die Ladeschaltung der Batterie wurde wie bereits im Kapitel \ref{sec:Energiespeicher} beim Abschnitt Schutzeinrichtungen ein Lade-IC verwendet. Dieser reguliert zuerst die Spannung wobei nach Erreichung des Schwellenwertes von 4.2V den Strom auf 0A herunter reguliert. Dieser Vorgang wurde während einem gesamten Ladevorgang der Batterie beobachtet und dokumentiert. Die nachfolgende Abbildung \ref{fig:Ladekurve-LadeIC} zeigt die Regulierung der Spannung (blaue Kurve) wie auch die Regulierung des Strome (rote Kurve) in Abhängigkeit der Zeit.

\begin{figure}[H]
	\begin{center}
		\includegraphics[width=100mm]{data/Platzhalter.jpg}
		\caption[Ladekurve Lade-IC]{Ladekurve Lade-IC} %picture caption
		\label{fig:Ladekurve-LadeIC}
	\end{center}
\end{figure}


\subsubsection*{induktive Ladeschaltung}\label{sec:batterie}
In diesem Abschnitt werden die Ergebnisse der induktiven Ladeschaltung präsentiert. Nachfolgend zeigt Abbildung \ref{fig:InduzierterStrom} die Abhängigkeit zwischen induziertem Strom und der Distanz z zwischen den Induktionsspulen.

\begin{figure}[H]
	\begin{center}
		\includegraphics[width=100mm]{data/Platzhalter.jpg}
		\caption[Induzierter Strom in Abhängigkeit der Distanz]{Induzierter Strom in Abhängigkeit der Distanz} %picture caption
		\label{fig:InduzierterStrom}
	\end{center}
\end{figure}

Eindrücklich ist hierbei dass zum Abstand der Primärspule, der Strom (lineaer, quadratisch, was auch immer) abnimmt. Aufgrund dieser Erkenntnis sind wir auch gezwungen eine möglichst kurze Entfernung zwischen den Spulen einzuhalten. Wir verwenden hierbei einen Abstand von rund 1.5mm, welcher auch der Materialdicke des Dojos entspricht. Es resultiert also ein an die Sekundärspule gelieferter Strom von rund 50mA.

\subsection{Testkonzept Software}\label{sec:testkonzeptSoftware}

\newpage
\section{Schlusswort} \label{sec:schlusswort}

Der Dōjō wurde ursprünglich vom Auftraggeber als kleiner hohler Stab konzeptioniert. In der ersten Entwicklungsphase wurde ein Prototyp realisiert, welcher optisch an und für sich nichts mit dem Dōjō zu tun hat, sondern lediglich die Funktionalität der Elektronik im Vordergrund steht. Der grosszügige Aufbau hat den Vorteil, dass einzelne Teilsysteme getestet und in einem weiteren Schritt zu einem Gesamtsystem zusammenbauen lassen. Diese Vorgehensweise ermöglichte eine saubere und effiziente Arbeitstrennung im Projektteam. Die einzelnen Komponenten wurden dabei alle so klein gewählt, dass sie ohne weiteres im Innenraum des Dōjōs platziert werden können. Ein bereits vorgefertigtes PCB-Layout stellt dabei ein passendes Gesamtkonstrukt dar. Die meisten Teilsysteme funktionieren wunschgemäss, wie zum Beispiel die SD-Karte. Sie wurde zuerst als entkoppeltes System verwendet. Der Zugriff auf den Speicher funktioniert einwandfrei und das Lesen und Schreiben der Audio-Files und der Audio-Namen funktioniert gemäss den Erwartungen. Weiter bietet der vorhandene Speicher von 8 GB genügend Platz für das Implementieren von mehreren Sprachen. Durch das Verzichten auf eine USB-Schnittstelle besteht keine Möglichkeit die SD-Karte ohne weiteres zu aktualisieren, oder den Inhalt im Allgemeinen zu verändern. Daher muss die SD-Karte manuell entnommen und über einen extra dafür vorgesehenen SD-Karten-Hub aktualisiert werden. Dadurch lassen sich mehrere SD-Karten gleichzeitig bearbeiten, wobei eine signifikanten Zeiteinsparung für die Betreiber erzielt wird. Weiter ist die SD-Karte im Gehäuse auch so angebracht, dass eine einfache Entnahme ohne Werkzeug oder Ähnliches möglich ist.
Es wäre theoretisch möglich gewesen eine USB-Schnittstelle für die Datenübertragung und auch Energieversorgung zu implementieren. Man musste sich allerdings auch die Frage stellen ob es dann noch Sinn macht, den Dōjō überhaupt induktiv zu laden. Beide Eigenschaften sind unmittelbar miteinander verknüpft, wobei die Entscheidung schlussendlich auf die induktive Ladung fiel. Der zentrale NRF52-Mikrocontroller übernimmt die Steuerung und kann die Bluetooth-Beacons einwandfrei detektieren, was über die Software umgesetzt wird. Mit einem geeigneten Algorithmus wird jeweils das Beacon mit dem stärksten Signal als das {\glqq korrekte\grqq} Kunstobjekt identifiziert. Das integrierte Bluetooth-Modul ermöglicht das Empfangen der entsprechenden Beacon-ID und kann diese über eine geeignete Funktion in eine Hexadezimal-Zahl wandeln. Anschliessend wird mit einer ID-Liste auf der SD-Karte verglichen und bei Übereinstimmung das entsprechende Audio-File abgespielt. Die implementierte Batterie versorgt alle eingebauten Geräte mit genügend Energie und ermöglicht eine Laufzeit von rund 6 Stunden. Die Ladung der Batterie erfolgt mittels induktiver Ladung. Aus diesem Grund wurde in Absprache mit dem Auftraggebers auf die USB-Verbindung verzichtet. Die induktive Ladung funktioniert einwandfrei und ermöglicht eine Stromübertragung von maximal 70mA. Somit können wir sagen, dass das Ziele gemäss Pflichtenheft erfüllt wurde. Das Abspielen der Audiodatei funktioniert, solange es eine Datei vom Typ WAV Unsigned 8-bit PCM ist. Im Pflichtenheft wurde definiert, dass maximal 214mW RMS am Knochenschallaktor vorhanden sind. Im Prototyp sind es nun 649.25mW RMS. Dies übersteigt den definierten Wert, ist aber dennoch gut. Der Wert wurde definiert, um ein Audiosignal zu erhalten, welches sich nicht überschlägt. Da dies beim Prototyp mit 650mW RMS Leistung am Knochenschallaktor so ist, wurde das Ziel trotzdem erreicht.\\
Alles in allem wurde ein Prototyp erreicht, welcher im groben alle Anforderungen an Hardware und Software erfüllt. Trotz anfänglicher Skepsis haben wir es geschafft, die Audioausgabe mittels PWM zu erzielen und konnten dadurch auf einen Audio-Chip verzichten. Weiter sind benutzerfreundliche Bedienungen wie zum Beispiel die Auswahl der Sprache eines der Highlights unseres Produktes. Ebenso muss die induktive Ladung erwähnt werden, welche einen unkomplizierten und bequemen Ladezyklus ermöglicht.
 
Obwohl ein funktionaler Prototyp vorliegt, gibt es noch diverse nicht-implementierte Schnittstellen. Zum Beispiel konnte keine Lautstärkeregelung realisiert werden. Da aber zu viel Zeit für die Audioausgabe aufgewendet wurde, blieb keine Zeit mehr für die Regulierung der Lautstärke. Eine intelligente Ausgabe der {\glqq Like-Liste\grqq} wurde ebenfalls nicht umgesetzt. Zur Zeit muss dafür die SD-Karte entfernt werden. Auch muss hier erwähnt werden, dass keines der im Pflichtenheft definierten Wunschziele umgesetzt wurde. Der Fokus wurde auf die Sollziele gelegt. Das Zusammenspiel zwischen Batterie und Mikrocontroller fehlt komplett. Eine Überwachung der Batterie ist nicht realisiert worden. Ebenfalls wurde keine Ladestationserkennung eingebaut. Weiter enthält die Software Code-Stellen, die nicht sehr flexibel aufgebaut sind. Dies sind vor allem die Switch-Case Abfragen, welche das Programm statisch machen. Eine unangenehme Eigenschaft der Bluetooth-Beacons ist das Stören der Audioausgabe durch ein {\glqq Knacken\grqq}.
 
Eine Weiterentwicklung des aktuellen Konzepts ist durchaus sinnvoll. Nachdem die Basis des Dōjō umgesetzt und getestet wurde, ist viel technisches \glqq Know-How \grqq vorhanden. Nachfolgend werden mögliche Optimierungs- und Weiterentwicklungsmöglichkeiten aufgelistet. 
Obwohl die induktive Energieübertragung einwandfrei funktioniert, lohnt es sich in einem weiteren Schritt diese noch zu verbessern. Für nachfolgende Weiterentwicklungen der Ladestufe empfiehlt es sich, einen Lade-IC zu verwenden, welcher eine grössere Eingangsspannung verarbeiten kann. Die Verwendung des LT1512 hätte den Vorteil, dass die Eingangsspannung zwischen $2.7V - 25V$ variieren kann. Die grosse Leistungsfähigkeit dieses Bauteils hat zwar zur Folge dass die Grösse des Bauteils $(L \times B \times H)$ mit $(6.2\times 5\times 1.7)mm$ eher gross ausfällt, sein Vorteil mit dem hohen Versorgungsspannungsbereich überwiegt jedoch. Des Weiteren würde es sich lohnen die Spulen neu auszulegen. Die eingebauten Flachspulen sind zwar platzsparend, kommen jedoch aufgrund ihrer Bauform ziemlich schnell an ihr Leistungsmaximum. Es würden sich somit Spulen empfehlen, wie sie auch in modernen elektrischen Zahnbürsten zum Einsatz kommen. Hierbei handelt es sich um zwei unterschiedlich grosse Ringspulen, von welchem die kleinere Tranceiver-Spule in der Ladeschaltung in die grössere Receiverspule im Dōjō gesteckt wird. Diese Ringspulen wären zwar ein wenig grösser, doch genau durch diesen Faktor könnte die Ladeschaltung mit grösseren Strömen betrieben und damit auch der Dōjō in kürzerer Zeit geladen werden.\\
Bei der Audioausgabe könnte die Qualität des Audiosignals verbessert werden. Hierbei wäre es sinnvoll ein anderes Dateiformat zu wählen. Dabei gibt es verschiedene Möglichkeiten. Unser Vorschlag wäre ein WAV (Microsoft) signed 16-bit PCM zu verwenden.\\
Weiter könnte eine Aktualisierung der SD-Karte über ein drahtloses Netzwerk wie z.B Bluetooth oder W-LAN umgesetzt werden. Dadurch liesse sich jeglicher Kontakt mit der SD-Karte durch einen Mitarbeiter vermeiden. Fraglich ist an dieser Stelle jedoch in welchem Umfang Aktualisierungen vorgenommen werden, da die drahtlose Datenübertragung nicht unbegrenzt Daten übermitteln kann. Diese drahtlose Übertragung würde auch das kluge auslesen der \glqq Like-Liste\grqq ermöglichen.\\
Eine weitere Möglichkeit ist die Umsetzung eines integrierten Tickets. Die automatische Zutrittsberechtigung zu verschiedenen Ausstellungsbereiche, ermöglicht einen individuellen Museumsbesuch. Die Software dazu, ist jedoch im Moment noch nicht für dieses Szenario ausgelegt und benötigt im Falle einer Implementation noch einiges an Aufwand.\\
Mit weitaus geringerem Aufwand könnte die Signalisation mittels LED Leiste implementiert werden. Im Moment wird noch eine LED auf dem Evaluations-Board angesteuert. Das Ansprechen einer RGB Leuchtdiode ist softwaretechnisch jedoch bereits programmiert und müsste für die Funktionalität nur noch im Dōjō eingebaut werden.
\newpage
\section{Ehrlichkeitserkl\"arung}
Der Projektleiter bestätigt mit der Unterschrift, dass der Bericht selbst verfasst und alle Quellen sauber und korrekt deklariert wurden.\\



\begin{tabular}{lp{20em}l}
 \hspace{3cm}   && \hspace{3cm} \\\cline{1-1}\cline{3-3}
 Ort, Datum     && Unterschrift Projektleiter
\end{tabular}

\newpage
\bibliographystyle{IEEEtran}
\bibliography{quelle}
\newpage
\listoffigures
\newpage
\begin{appendix}
\section{Anhang}\label{sec:anhang}

\subsection{Messresultate}\label{sec:messresultate}

\subsection{Lizenztexte}\label{sec:lizenztexte}

\begin{figure}[H]
	\begin{center}
		\includegraphics[width=0.7\textwidth]{data/Lizenztext_FatFs.png}
		\caption{Lizenztext FatFs library} %picture caption
		\label{fig:Lizentext FatFs}
	\end{center}
\end{figure}

\end{appendix}

%%%%%%%%%%%%%%%%%%%%%%%%%%%%%%%%%%%%%%%%%%%%%%%%%%%%%%%%
\end{document}

