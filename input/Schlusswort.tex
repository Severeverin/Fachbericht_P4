\section{Schlusswort} \label{sec:schlusswort}

Dieses Kapitel besteht noch aus einzelnen Argumenten und Abschnitten und ist noch nicht miteinander verknüpft

\textbf{a) Was läuft?Was wurde im Projekt erreicht? Welche in der Aufgabenstellung gestellten Anforderungen wurden realisiert? Wo wurden diese übertroffen?}
Der Dōjō wurde ursprünglich in einem kleinen hohlen Stab konzeptioniert. In dieser ersten Entwicklungsphase wurde ein Prototyp realisiert, welcher optisch an und für sich nichts mit dem Dōjō zu tun hat, sondern lediglich die Funktionalität der Elektronik im Vordergrund steht. Der grosszügige Aufbau hat den Vorteil, dass sich einzelne Komponenten wie z.B. die Ladeschaltung für die Energiespeicherung, Mikrocontroller auf einem Evaluation Board oder der Knochenschallaktor als einzelne Teilsystem getestet und in einem weiteren Schritt zu einem Gesamtsystem zusammenbauen lassen. Diese Vorgehensweise ermöglichte eine saubere und effiziente Arbeitstrennung. Nach dem alle Komponenten dimensioniert und getestet wurden, konnte ein fertiges in den Dojo passendes PCB Design erstellt werden, welches in einem weiteren Schritt in den Dōjō eingebaut werden könnte.
 
Das Gesamtsystem umfasst hierbei die Hardwarekomponenten wie induktive Ladeschaltung, Batterie, Microcontroller, sowie notwendige Komponenten für die Audioausgabe. Die Software beinhaltet alle programmierbaren Ebenen wie die State Machine, aber auch die Audio PWM Ausgabe, das Bluetooth und die Programmierung des Nordic SDK.
Die Erkennung der Bluetooth-Beacons funktioniert einwandfrei, solange nicht mehr als 10 Beacons in unmittelbarer Nähe sind. Das Detektieren erfolgt über die Software. Mit einem geeigneten Algorithmus wird jeweils das Beacon mit dem stärksten Signal als das\glqq korrekte\grqq Kunstobjekt identifiziert. Das integrierte Bluetooth-Modul im Mikrocontroller ermöglicht das Empfangen der entsprechenden Beacon-ID und kann diese über eine geeignete Funktion in eine Hexadezimal-Zahl wandeln. Anschliessend wird mit einer ID-Liste auf der SD-Karte verglichen und bei Übereinstimmung das entsprechende Audio-File abgespielt.\\
Die induktive Ladung funktioniert einwandfrei und ermöglicht eine Stromübertragung von maximal 70mA. Für nachfolgende Weiterentwicklungen der Ladestufe empfiehlt es sich, einen Lade-IC zu verwenden, welcher eine grössere Eingangsspannung verarbeiten kann. Die Verwendung eines LT1512 hat den Vorteil, dass die Eingangsspannung zwischen $2.7V - 25V$ variieren kann. Dieser Grosse Spannungsbereich hat jedoch zur Folge, dass die Grösse des Bauteils $(L \times B \times H)$ mit $(6.2\times 5\times 1.7)mm$ eher gross ausfällt. Trotz seiner grossen Bauform überwiegt hierbei aber der Vorteil mit der hohen Versorgungsspannung.\\
Desweiteren würde es sich Lohnen die Spulen zu optimieren. Die eingebauten Flachspulen sind zwar platzsparen, kommen jedoch aufgrund ihrer Bauform ziemlich schnell an ihr Leistungsmaximum. Hier würden sich solche Spulen sich empfehlen, wie sie z.B. auch in modernen elektrischen Zahnbürsten verwendet werden. Hierbei handelt es sich um zwei unterschiedlich grosse Ringspulen, von welchem die kleinere Tranceiver-Spule in der Ladeschaltung in die grössere Receiverspule  im Dojo gesteckt wird. Diese Ringspulen wären zwar ein wenig grösser, doch durch die grössere Bauform könnte die Ladeschaltung mit grösseren Strömen betrieben und damit auch der Dojo noch schneller geladen werden.\\
Mit der Verwendung des komplexen und leistungsfähigen Mikrocontrollers NRF52832 können alle Prozesse zentral gesteuert werden und dadurch entfallen weitere Controller. Ebenfalls kann dadurch der Energieverbrauch gesenkt werden und die Verwendung eines zusätzlichen Bluetooth-Moduls entfällt.
 
\textbf{b) Was läuft nicht?Wo besteht Verbesserungsbedarf? Was konnte nicht realisiert werden? Hier ist eine kurze Angabe der Gründe sinnvoll.}
Mit dem Verzicht einer USB-Schnittstelle besteht keine Möglichkeit die SD-Karte ohne weiteres zu aktualisieren oder den Inhalt im Allgemeinen zu verändern. Daher sieht das Konzept vor, die SD-Karte manuell zu entnehmen und dann über einen extra dafür vorgesehenen SD-Karten-Hub zu aktualisieren. Dadurch lassen sich mehrere SD-Karten gleichzeitig bearbeiten und führt damit zu einer signifikanten Zeiteinsparung für die Betreiber. Weiter wird die SD-Karte im Gehäuse auch so angebracht, dass eine einfache Entnahme ohne Werkzeug oder Ähnliches möglich ist. Theoretisch wäre es möglich gewesen eine USB-Schnittstelle zu implementieren. Allerdings muss man sich dann auch unmittelbar die Frage stellen ob es dann noch Sinn macht, den Dōjō induktiv zu laden. Beide Eigenschaften sind unmittelbar miteinander verknüpft. Die Entscheidung fiel dann auf die induktive Ladung der Anwendung.
 
\textbf{c) Optimierungs- / Weiterentwicklungsmöglichkeiten: Welche Optimierungsmöglichkeiten bestehen? Was könnte wie besser gemacht werden? Was muss bei einer Weiterentwicklung bedacht werden?}
Eine Weiterentwicklung des aktuellen Konzepts ist durchaus sinnvoll. Nachdem die Basis des Dōjō umgesetzt und getestet wurde, ist viel technisches \glqq Know-How \grqq vorhanden. Mögliche Weiterentwicklungsmöglichkeiten sind nachfolgend aufgelistet.\\
Einerseits kann eine Aktualisierung der SD-Karte über ein drahtloses Netzwerk wie z.B Bluetooth oder W-LAN umgesetzt werden. Dadurch liesse sich jeglicher Kontakt mit der SD-Karte durch einen Mitarbeiter vermeiden. Fraglich ist an dieser Stelle jedoch in welchem Umfang Aktualisierungen vorgenommen werden, da die drahtlose Datenübertragung nicht unbegrenzt Daten übermitteln kann.\\
Eine weitere Möglichkeit ist die Umsetzung eines integrierten Tickets. Dieses wurde im Verlaufe des Berichts weder erläutert noch umgesetzt. Die automatische Zutrittsberechtigung zu verschiedenen Ausstellungsbereiche, ermöglicht einen individuellen Museumsbesuch. Die Software dazu, ist jedoch im Moment noch nicht für dieses Szenario ausgelegt und benötigt bei einer Implementation noch einiges an Aufwand.\\
Ein weiterer Punkt welcher mit weitaus geringerem Aufwand implementiert werden könnte, ist die Signalisation mittels LED Leiste. Im Moment wird noch eine LED ohne rgb angesteuert. Das Ansprechen einer RGB Leuchtdiode ist softwaretechnisch jedoch programmiert und müsste für die Funktionalität nur noch im Dojo eingebaut werden. 