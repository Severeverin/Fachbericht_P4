\section{Schlusswort} \label{sec:schlusswort}

Dieses Kapitel besteht noch aus einzelnen Argumenten und Abschnitten und ist noch nicht miteinander verknüpft

\textbf{a) Was läuft?Was wurde im Projekt erreicht? Welche in der Aufgabenstellung gestellten Anforderungen wurden realisiert? Wo wurden diese übertroffen?}

Dōjō wurde ursprünglich in einem kleinen hohlen Stab konzeptioniert. In einer ersten Entwicklungsphase wurde ein Prototyp realisiert, welcher optisch an und für sich nichts mit dem Dōjō zu tun hat. Das hat den Vorteil, dass sich einzelne Komponenten wie z.B. eine Ladeschaltung für die Energiespeicherung, Mikrocontroller Evaluation Board oder der Knochenschallaktor als einzelne Teilsystem einfach zu einem Gesamtsystem zusammensetzen lassen. Dadurch lassen sich die Arbeiten voneinander entkoppeln und Messungen können mit deutlich weniger Aufwand durchgeführt werden. Das Evaluation Board des Mikrocontrollers bringt den grössten Vorteil mit sich. Steckplätze für Verbindungen und weitere Komponenten sind bereits vorhanden und können ohne weiteres genutzt werden. Ebenfalls kann dadurch die Programmierung der einzelnen Teilbereiche durch das Verwenden mehrerer Evaluation Boards unabhängig voneinander durchgeführt und getestet werden. Zusätzlich wurde dabei darauf geachtet, dass sich die einzelnen Komponenten auf einem kleinen PCB implementieren lassen, das sich dann wiederum in den Dōjō einbauen lässt.

Die Erkennung der Bluetooth-Beacons funktioniert einwandfrei, solange nicht mehr als 10 Beacons in unmittelbarer Nähe sind. Das Detektieren erfolgt über die Software. Mit einem geeigneten Algorithmus wird jeweils das Beacon mit dem stärksten Signal als das\glqq korrekte\grqq Kunstobjekt identifiziert. Das integrierte Bluetooth-Modul im Mikrocontroller ermöglicht das Empfangen der entsprechenden Beacon-ID und kann diese über eine geeignete Funktion in eine Hexadezimal-Zahl wandeln. Anschliessend wird mit einer ID-Liste auf der SD-Karte verglichen und bei Übereinstimmung das entsprechende Audio-File abgespielt.

Die Energieversorgung wird über kleine Batterien umgesetzt, die im Bodenbereich des Dōjō angebracht werden. Das spezielle daran ist, dass sie induktiv geladen werden. Dadurch kann jeglicher \glqq Kabelsalat\grqq zwischen dem Gerät und Ladestation vermieden werden. Ein kleine Einbuchtung am Boden des Dōjō ermöglicht das einfache aufstecken und laden. Aus diesem Grund wurde auf eine Implementierung einer USB-Connection verzichtet.

Like-Button

Mit der Verwendung des komplexen und leistungsfähigen Mikrocontrollers NRF52832 können alle Prozesse zentral gesteuert werden und dadurch entfallen weitere Controller. Ebenfalls kann dadurch der Energieverbrauch gesenkt werden und die Verwendung eines zusätzlichen Bluetooth-Moduls entfällt.

\textbf{b) Was läuft nicht?Wo besteht Verbesserungsbedarf? Was konnte nicht realisiert werden? Hier ist eine kurze Angabe der Gründe sinnvoll.}

Mit dem Verzicht einer USB-Schnittstelle besteht keine Möglichkeit die SD-Karte ohne weiteres zu aktualisieren oder den Inhalt im Allgemeinen zu verändern. Daher sieht das Konzept vor, die SD-Karte manuell zu entnehmen und dann über einen extra dafür vorgesehenen SD-Karten-Hub zu aktualisieren. Dadurch lassen sich mehrere SD-Karten gleichzeitig bearbeiten und führt damit zu einer signifikanten Zeiteinsparung für die Betreiber. Weiter wird die SD-Karte im Gehäuse auch so angebracht, dass eine einfache Entnahme ohne Werkzeug oder Ähnliches möglich ist. Theoretisch wäre es möglich gewesen eine USB-Schnittstelle zu implementieren. Allerdings muss man sich dann auch unmittelbar die Frage stellen ob es dann noch Sinn macht, den Dōjō induktiv zu laden. Beide Eigenschaften sind unmittelbar miteinander verknüpft. Die Entscheidung fiel dann auf die induktive Ladung der Anwendung.

\textbf{c) Optimierungs- / Weiterentwicklungsmöglichkeiten: Welche Optimierungsmöglichkeiten bestehen? Was könnte wie besser gemacht werden? Was muss bei einer Weiterentwicklung bedacht werden?
}

Eine Weiterentwicklung des aktuellen Konzepts ist durchaus sinnvoll. Nachdem die Basis des Dōjō umgesetzt und getestet wurde, ist viel technisches \glqq Know-How \grqq vorhanden. Einerseits kann eine Aktualisierung der SD-Karte über ein drahtloses Netzwerk wie z.B Bluetooth oder W-LAN umgesetzt werden. Dadurch liesse sich jedglicher Kontakt mit der SD-Karte durch einen Mitarbeiter vermeiden. Fraglich ist an dieser Stelle jedoch in welchem Umfang Aktualisierungen vorgenommen werden, da die drahtlose Datenübertragung nicht unbegrenzt Daten übermitteln kann.

Ein weitere Möglichkeit ist die Umsetzung der Sprachwahl. Konzeptionell wurde diese im Verlaufe des Berichts erläutert aber nicht umgesetzt. Durch die automatische Konfiguration braucht es keinen Mitarbeiter, der die jeweilige Sprache konfiguriert. Softwaremässig kann dies mit bescheidenem Aufwand umgesetzt werden. Dabei werden alles Sprachen auf der SD-Karte platziert und mit einer entsprechenden Endung, wie z.B en, de oder it versehen. Durch die zugehörige Beacon-ID kann dann jeweils in allen Suchalgorithmen die Namensgebung hinzugefügt werden. Somit wird dann automatisch die richtige Audio-Datei mit der richtigen Sprache ausgegeben.

Like Button mit Netz.