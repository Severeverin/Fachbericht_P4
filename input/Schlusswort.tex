\section{Schlusswort} \label{sec:schlusswort}

Der Dōjō wurde ursprünglich vom Auftraggeber als kleiner hohler Stab konzeptioniert. In dieser ersten Entwicklungsphase wurde ein Prototyp realisiert, welcher optisch an und für sich nichts mit dem Dōjō zu tun hat, sondern lediglich die Funktionalität der Elektronik im Vordergrund steht. Der grosszügige Aufbau hat den Vorteil, dass einzelne Teilsysteme getestet und in einem weiteren Schritt zu einem Gesamtsystem zusammenbauen lassen. Diese Vorgehensweise ermöglichte eine saubere und effiziente Arbeitstrennung im Projektteam. Die einzelnen Komponenten wurden dabei alle so klein gewählt, dass sie ohne weiteres im Innenraum des Dōjōs platziert werden können. Ein bereits vorgefertigtes PCB-Layout stellt dabei ein passendes Gesamtkonstrukt dar.

Die SD-Karte wurde zuerst als entkoppeltes System verwendet. Der Zugriff auf den Speicher funktioniert einwandfrei und das Lesen und Schreiben der Audio-Files und der Audio-Namen funktioniert gemäss den Erwartungen. Weiter bietet der vorhandene Speicher von 8 GB genügend Platz für das Implementieren von mehreren Sprachen. Durch das Verzichten auf eine USB-Schnittstelle besteht keine Möglichkeit die SD-Karte ohne weiteres zu aktualisieren, oder den Inhalt im Allgemeinen zu verändern. Daher muss die SD-Karte manuell entnommen und über einen extra dafür vorgesehenen SD-Karten-Hub aktualisiert werden. Dadurch lassen sich mehrere SD-Karten gleichzeitig bearbeiten, wobei eine signifikanten Zeiteinsparung für die Betreiber erzielt wird. Weiter ist die SD-Karte im Gehäuse auch so angebracht, dass eine einfache Entnahme ohne Werkzeug oder Ähnliches möglich ist.
Es wäre theoretisch möglich gewesen eine USB-Schnittstelle für die Datenübertragung und auch Energieversorgung zu implementieren. Man musste sich allerdings auch die Frage stellen ob es dann noch Sinn macht, den Dōjō überhaupt induktiv zu laden. Beide Eigenschaften sind unmittelbar miteinander verknüpft, wobei die Entscheidung schlussendlich auf die induktive Ladung fiel. Der zentrale NRF52-Mikrocontroller übernimmt die Steuerung und kann die Bluetooth-Beacons einwandfrei detektieren, was über die Software umgesetzt wird. Mit einem geeigneten Algorithmus wird jeweils das Beacon mit dem stärksten Signal als das{\glqq korrekte\grqq} Kunstobjekt identifiziert. Das integrierte Bluetooth-Modul ermöglicht das Empfangen der entsprechenden Beacon-ID und kann diese über eine geeignete Funktion in eine Hexadezimal-Zahl wandeln. Anschliessend wird mit einer ID-Liste auf der SD-Karte verglichen und bei Übereinstimmung das entsprechende Audio-File abgespielt. Die implementierte Batterie versorgt alle eingebauten Geräte mit genügend Energie und ermöglicht eine Laufzeit von rund 6 Stunden. Die Ladung der Batterie erfolgt mittels induktiver Ladung. Aus diesem Grund wurde in Absprache mit dem Auftraggebers auf die USB-Verbindung verzichtet. Die induktive Ladung funktioniert einwandfrei und ermöglicht eine Stromübertragung von maximal 70mA. Somit können wir sagen, dass das Ziele gemäss Pflichtenheft erfüllt wurde. Das Abspielen der Audiodatei funktioniert, solange es eine Datei vom Typ WAV Unsigned 8-bit PCM ist. Im Pflichtenheft wurde definiert, dass maximal 214mW RMS am Knochenschallaktor vorhanden sind. Im Prototyp sind es nun 649.25mW RMS. Dies übersteigt den definierten Wert, ist aber dennoch gut. Der Wert wurde definiert, um ein Audiosignal zu erhalten, welches sich nicht überschlägt. Da dies beim Prototyp mit 650mW RMS Leistung am Knochenschallaktor so ist, wurde das Ziel trotzdem erreicht.
 
Alles in allem wurde ein Prototyp erreicht, welcher im groben alle Anforderungen an Hardware und Software erfüllt. Trotz anfänglicher Skepsis haben wir es gschafft, die Audioausgabe mittels PWM zu erzielen und konnten dadurch auf einen Audio-Chip verzichten. Weiter sind benutzerfreundliche Bedienungen wie zum Beispiel die Auswahl der Sprache eines der Highlights unseres Produktes. Ebenso muss die induktive Ladung erwähnt werden, welche einen unkomplizierten und bequemen Ladezyklus ermöglicht.

Obwohl ein funktionaler Prototyp vorliegt, gibt es noch diverse nicht-implementierte Schnittstellen. Zum Beispiel konnte keine Lautstärkeregelung realisiert werden. Da aber zu viel Zeit für die Audioausgabe aufgewendet wurde, blieb keine Zeit mehr für die Regulierung der Lautstärke. Eine intelligente Ausgabe der {\glqq Like-Liste\grqq} wurde ebenfalls nicht umgesetzt. Zur Zeit muss dafür die SD-Karte entfernt werden. Auch muss hier erwähnt werden, dass keines der im Pflichtenheft definierten Wunschziele umgesetzt wurde. Der Fokus wurde auf die Sollziele gelegt. Das Zusammenspiel zwischen Batterie und Mikrocontroller fehlt komplett. Eine Überwachung der Batterie ist nicht realisiert worden. Ebenfalls wurde keine Ladestationserkennung eingebaut. Weiter enthält die Software Code-Stellen, die nicht sehr flexibel aufgebaut sind. Dies sind vor allem die Switch-Case Abfragen, welche das Programm statisch machen. Eine unangenehme Eigenschaft der Bluetooth-Beacons ist das Stören der Audioausgabe durch ein {\glqq Knacken\grqq}.
Eine Weiterentwicklung des aktuellen Konzepts ist durchaus sinnvoll. Nachdem die Basis des Dōjō umgesetzt und getestet wurde, ist viel technisches \glqq Know-How \grqq vorhanden. Nachfolgend werden mögliche Optimierungs- und Weiterentwicklungsmöglichkeiten aufgelistet. 

Obwohl die induktive Energieübertragung einwandfrei funktioniert, lohnt es sich in einem weiteren Schritt diese noch zu verbessern. Für nachfolgende Weiterentwicklungen der Ladestufe empfiehlt es sich, einen Lade-IC zu verwenden, welcher eine grössere Eingangsspannung verarbeiten kann. Die Verwendung des LT1512 hätte den Vorteil, dass die Eingangsspannung zwischen $2.7V - 25V$ variieren kann. Die grosse Leistungsfähigkeit dieses Bauteils hat zwar zur Folge dass die Grösse des Bauteils $(L \times B \times H)$ mit $(6.2\times 5\times 1.7)mm$ eher gross ausfällt, sein Vorteil mit dem hohen Versorgungsspannungsbereich überwiegt jedoch. Desweiteren würde es sich lohnen die Spulen zu optimieren. Die eingebauten Flachspulen sind zwar platzsparend, kommen jedoch aufgrund ihrer Bauform ziemlich schnell an ihr Leistungsmaximum. Hier würden sich solche Spulen sich empfehlen, wie sie z.B. auch in modernen elektrischen Zahnbürsten verwendet werden. Hierbei handelt es sich um zwei unterschiedlich grosse Ringspulen, von welchem die kleinere Tranceiver-Spule in der Ladeschaltung in die grössere Receiverspule  im Dōjō gesteckt wird. Diese Ringspulen wären zwar ein wenig grösser, doch durch die grössere Bauform könnte die Ladeschaltung mit grösseren Strömen betrieben und damit auch der Dōjō noch schneller geladen werden.\\
Bei der Audioausgabe könnte die Qualität des Audiosignals verbessert werden. Hierbei wäre es sinnvoll ein anderes Dateiformat zu wählen. Dabei gibt es verschiedene Möglichkeiten. Unser Vorschlag wäre ein WAV (Microsoft) signed 16-bit PCM zu verwenden.\\
Einerseits kann eine Aktualisierung der SD-Karte über ein drahtloses Netzwerk wie z.B Bluetooth oder W-LAN umgesetzt werden. Dadurch liesse sich jeglicher Kontakt mit der SD-Karte durch einen Mitarbeiter vermeiden. Fraglich ist an dieser Stelle jedoch in welchem Umfang Aktualisierungen vorgenommen werden, da die drahtlose Datenübertragung nicht unbegrenzt Daten übermitteln kann. Diese drahtlose Übertragung würde auch das kluge auslesen der like-Liste ermöglichen.\\
Eine weitere Möglichkeit ist die Umsetzung eines integrierten Tickets. Dieses wurde im Verlaufe des Berichts weder erläutert noch umgesetzt. Die automatische Zutrittsberechtigung zu verschiedenen Ausstellungsbereiche, ermöglicht einen individuellen Museumsbesuch. Die Software dazu, ist jedoch im Moment noch nicht für dieses Szenario ausgelegt und benötigt bei einer Implementation noch einiges an Aufwand.\\
Mit weitaus geringerem Aufwand könnte die Signalisation mittels LED Leiste implementiert werden. Im Moment wird noch eine gelbe LED angesteuert. Das Ansprechen einer RGB Leuchtdiode ist softwaretechnisch jedoch bereits programmiert und müsste für die Funktionalität nur noch im Dōjō eingebaut werden.