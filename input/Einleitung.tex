\section{Einleitung}\label{sec:einleitung}

Museen bieten die Möglichkeit unterschiedlichste Ausstellungsobjekte unter einem Dach zu betrachten. Die Art der ausgestellten Kunst ist hier von Austellung zu Austellung unterschiedlich, was jedoch bleibt ist die kreative Wahrnehmung der Besucher. Wohl nirgends kann man so gut in seinen eigenen Gedanken versinken und sich Gedanken über ein Ausstellungsobjekt machen wie in einem Museum. Um Besucher anzulocken, sind Museen auf Innovation angewiesen, welche zum einen die Übergabe von Informationen möglichst benutzerfreundlich gestaltet, aber auch ein angenehmes Ambiente schafft. Hierbei kommt auch vermehrt der Einsatz von Smartphones zum Zuge, wobei die Problematik darin besteht, dass man der Aussenwelt gefährlich nahe kommt und dadurch abgelenkt wird. Um diesem Problem entgegenzuwirken, wird ein von Frau J. Kalbermatter designter Audioguide namens Dōjō entwickelt, welcher Kunstobjekte drahtlos erkennen und darüber gespeicherte Informationen via Körperschalltechnik zum Museumsbesucher bringen kann. Durch eine Vielzahl von weiteren Funktionen ist ein neuer Informationsaustausch an den Museumsbesucher möglich, und Ablenkungen zur Aussenwelt sind trotz modernster Technik in weiter Ferne.

Ziel im Projekt 4, Studiengang Elektro- und Informationstechnik an der Fachhochschule Nordwestschweiz, war es das funktionelle Konzept von Frau Kalbermatter durch die Verwendung von elektrotechnischen Bauteilen zu realisieren. Dazu wurde die drahtlose Erkennung der Kunstobjekte mittels Bluetooth Low Energy (BLE) Beacons erreicht. Genannte Beacons müssen in unmittelbare Nähe der Kunstobjekte angebracht sein. Die Informationen zu den Kunstobjekten wurden als Audiofiles auf einer herausnehmbaren SD-Karte gespeichert und werden zum Abspielen via PWM-Ausgang des Mikrokontrollers (nRF52832) über einen Klasse D Verstärker auf den Knochenschallaktor gegeben. Tasten für die Wiedergabekontrolle (Start, Stopp) wurden implementiert sowie die erwähnte \glqq Like\grqq -Taste. Ausserdem verfügt der Dōjō über einen Li-Ionen-Akku, welcher induktiv geladen wird. Damit der Dōjō gänzlich drahtlos bleibt, erfolgen Datendownload und Konfiguration ebenso über Bluetooth. Der integrierte Mikrokontroller beinhaltet die Software und übernimmt somit die Erkennung, Ansteuerung und Koordination der Hardware.

Es wurde ein Prototyp der Elektronik realisiert. Die Ansteuerung der Audiofiles erfolgt über Bluetooth-Beacons, welche variabel bis zu einer Distanz von maximal 40m erkennt werden. Die eingebaute «Like»-Taste ermöglicht, favorisierte Kunstobjekte zu vermerken und die dazugehörigen Informationen am Ende des Besuches digital oder in Form einer Broschüre beim Ausgang als Erinnerung mitzunehmen. Ausserdem besitzt der Dōjō einen integrierten Akku mit einer Kapazität von 800 mAh, welcher bei pausenloser Audioausgabe genug Energie für rund 6 Stunden liefert. Die Induktionsladung lädt den Akku im Normalbetrieb zu 100\% innert 13 Stunden. Ausserdem sorgt ein Tiefenentladungsschutz der Batterie dafür, dass der Dōjō bei einer Betriebsnennspannung von unter 3V ausgeschaltet wird.

Der nachfolgende Bericht umfasst vier Hauptkapitel. Das erste Kapitel beschreibt das Gesamtkonzept, welches die gesamte Anwendung auslegt. Die nachfolgenden zwei Kapitel sind in Hardware und Software gegliedert. Am Schluss folgt noch das Kapitel Validierung, in welchem die einzelnen Systeme getestet und dokumentiert werden.

 