\section{Einleitung}\label{sec:einleitung}
Museen bieten die Möglichkeit unterschiedlichste Ausstellungsobjekte unter einem Dach zu betrachten. Die Art der ausgestellten Kunst ist hier von Ausstellung zu Ausstellung unterschiedlich, was jedoch beständig ist, ist eine spirituelle Wahrnehmung. Wohl nirgends kann man so gut in seinen eigenen Gedanken versinken und sich Gedanken über ein Ausstellungsobjekt machen wie bei einem Museum. Um Besucher anzulocken, sind Museen auf Innovation angewiesen, welche zum einen die Übergabe von Informationen möglichst Benutzerfreundlich gestaltet aber auch eine angenehme Ambiance schaffen. Hierbei kommt auch vermehrt der Einsatz von Smartphones zum Zuge, wobei die Problematik darin besteht, dass man der Aussenwelt gefährlich nahe kommt und dadurch abgelenkt wird.

Ziel dieses Projektes ist es eine smarte Lösung für einen Audio-Guide zu realisieren ohne dass das eigene Smartphone benötigt wird. Im Fokus stehen dabei, dass Informationen bequem zum Nutzer gelangen, ohne dass Ablenkungen unterschiedlichster Art einem zurück in den Alltag holen. Des Weiteren soll er das herkömmliche Zutrittsticket ablösen, wodurch das Museum die Möglichkeit hat, ihre Ausstellung in mehrere Bereiche zu unterteilen. Der Besucher kann dadurch beim Eintritt die auf ihn zugeschnittenen Bereiche auswählen und durch die integrierte Zutrittsberechtigung im Audio-Guide nur in die bezahlten Räume eintreten. Als Grundlage für das Design dient ein von der Auftraggeberin designter Museums Audio-Guide namens Dōjō. In dieser Projektarbeit ist das Ziel, einen funktionierenden Prototyp herzustellen, wodurch danach eine finale Version erstellt und in den Dōjō eingebaut werden könnte.

Der Dōjō ist vergleichbar mit einem runden Stab mit der Länge von XXX und XXX cm Durchmesser. Dieser weist sowohl eine Sprachausgabe mittels Knochenschallgeber, als auch weitere Peripherien wie einer simplen Audiosteuerung, Bluetooth-Beacon Erkennung und einem «Like Button». Sein simples Design und seine einfach Anwendung, ist für jede Altersgeneration geeignet. Die Realisierung dieses Audio-Guides erfolgt in einem Hardware- und einem Softwareteil. Der Hardwareteil wird hierbei durch die Ladeschaltung, die Energiespeicherung und Überwachung, wie auch der Audioausgabe bestimmt. Der Softwareteil übernimmt die Erkennung, Ansteuerung und Koordination der Hardware Komponenten. Die Bearbeitung erfolgt in dem jedes Teammitglied ein ihm zugeteilter Aufgabenbereich bearbeitet.

Für die Realisierung wird ein Prototyp entworfen, welcher zu XXX verschiedenen Sprachen XXX Stunden Audioausgabe speichern kann. Die Ansteuerung der Audiofiles erfolgt über Bluetooth-Beacon Erkennung, welche ab einer Distanz von XXX m erkennt werden. Der eingebaute «Like-Button» ermöglicht favorisierte Bilder abzuspeichern und diese am Ende des Besuches digital oder in Form einer Broschüre beim Ausgang als Erinnerung mitzunehmen.

Der nachfolgende Bericht ist durch drei Hauptbereiche definiert. Der erste Bereich umfasst das Gesamtkonzept, welcher die gesamte Anwendung auslegt. Die nachfolgenden zwei Hauptbereiche sind in Hardware und Software gegliedert. Die Hardware teil sich wiederum in die Themengebiete Energieübertragung, Energiespeicherung und Audioausgabe auf. Die Software beinhaltet die Unterbereiche der Bluetooth Kommunikation, Bluetooth-Beacon Erkennung, sowie die gesamte Programmstruktur des verwendeten Microcontrollers.
