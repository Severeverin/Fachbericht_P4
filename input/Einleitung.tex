\section{Einleitung}\label{sec:einleitung}
Zielloses Umherwandern und zu wenig Wissen über das Besondere an der Kunst sind nur einige Gründe weshalb bei einem Museumsbesuch oft der Besuch selbst mehr in Erinnerung bleibt als die eigentlichen Kunstobjekte. Um diesem Problem entgegenzuwirken, hat Frau Jana Kalbermatter das Gehäuse eines modernen Audioguides entwickelt und sich dazu ein funktionales Konzept ausgedacht. Dieser Audioguide, Dōjō genannt, soll Kunstobjekte drahtlos erkennen und darüber gespeicherte Informationen via Körperschalltechnik zum Museumsbesucher bringen. Dazu muss sich der Benutzer das Ende des stabförmigen Audioguides hinter das Ohr halten. Dank der Informationsübertragung via Körperschall und dem Verzicht auf Kopfhörer, erfüllt der Dōjō höchste Hygieneansprüche, was die Benutzung angenehmer macht. Weiter soll der Audioguide auch als Ticket fungieren und die Zutrittsberechtigung regeln. Das Drücken einer implementierten \glqq Like\grqq-Taste soll vermerken, welche Kunstobjekte dem Besucher gefallen haben. Das ermöglicht es, die Informationen über ein interessantes Kunstobjekt am Ende des Besuchs in Form einer Broschüre oder per E-Mail an den Besucher zu übergeben. Mit diesen Funktionen wird der Dōjō den Museumsbesuch hygienisch und angenehmer gestalten. Ausserdem wird er helfen, das Allgemeinwissen zu verbessern. Dadurch werden die Kunstobjekte und nicht der Besuch selbst zu einem unvergesslichen Erlebnis werden.\\
Ziel im Projekt 4 im Studiengang Elektro- und Informationstechnik an der Fachhochschule Nordwestschweiz war es deshalb, das funktionelle Konzept von Frau Kalbermatter durch die Verwendung von elektrotechnischen Bauteilen zu realisieren. Dazu wurde die drahtlose Erkennung der Kunstobjekte mittels Bluetooth Low Energy (BLE) Beacons erreicht. Genannte Beacons müssen in unmittelbare Nähe der Kunstobjekte angebracht sein. Die Informationen zu den Kunstobjekten wurden als Audiofiles auf einer herausnehmbaren SD-Karte gespeichert und werden zum Abspielen via PWM-Ausgang des Mikrokontrollers (nRF52832) über einen Klasse D Verstärker auf den Knochenschallaktor gegeben. Tasten für die Wiedergabekontrolle (Start, Stopp, Lauter, Leiser) wurden implementiert sowie die erwähnte \glqq Like\grqq -Taste. Ausserdem verfügt der Dōjō über einen Li-Ionen-Akku, welcher induktiv geladen wird. Damit der Dojo gänzlich drahtlos bleibt, erfolgen Datendownload und Konfiguration ebenso über Bluetooth. Der integrierte Mikrokontroller beinhaltet die Software und übernimmt somit die Erkennung, Ansteuerung und Koordination der Hardware.\\
Es wurde ein Prototyp der Elektronik realisiert. Er kann zu XXX verschiedenen Sprachen XXX Stunden Audioausgabe speichern. Die Ansteuerung der Audiofiles erfolgt über Bluetooth-Beacons, welche bis zu einer Distanz von XXX m erkennt werden. Die eingebaute «Like»-Taste ermöglicht, favorisierte Kunstobjekte zu vermerken und die dazugehörigen Informationen am Ende des Besuches digital oder in Form einer Broschüre beim Ausgang als Erinnerung mitzunehmen. Ausserdem besitzt der Dōjō einen integrierten Akku mit einer Kapazität von XXX mAh, welcher bei pausenloser Audioausgabe genug Energie für XXX Stunden liefert. Die Induktionsladung lädt den Akku zu 100\% innert XXX Stunden. Ausserdem sorgt ein Tiefenentladungsschutz dafür, dass der Dojo bei XXX\% Akkuladestand ausgeschaltet wird um den Akku vor Schäden zu bewahren.\\
Der nachfolgende Bericht umfasst drei Hauptbereiche. Der erste Bereich (Kapitel 2) umfasst das Gesamtkonzept, welcher die gesamte Anwendung auslegt. Die nachfolgenden zwei Hauptbereiche sind in Hardware (Kapitel 3) und Software (Kapitel 4) gegliedert. Die Hardware teil sich wiederum in die Themengebiete Energieübertragung (Kapitel 3.1), Energiespeicherung (Kapitel 3.2) und Audioausgabe über den Knochenschallaktor (Kapitel 3.5) auf. Die Software beinhaltet die Unterbereiche der State Machine (Kapitel 4.1), Bluetooth (Kapitel 4.3), sowie die gesamte Programmstruktur der SD-Karte (Kapitel 4.4) und Audioausgabe über PWM (Kapitel 4.5). In Kapitel 5 befindet sich die Validierung.\\
