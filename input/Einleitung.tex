\section{Einleitung}\label{sec:einleitung}

Zielloses Umherwandern und zu wenig Wissen darüber, was das Besondere an der Kunst ist, sind Gründe, weshalb bei einem Museumsbesuch oft der Besuch selbst mehr in Erinnerung bleibt als die Kunstobjekte. Um diesem Problem entgegenzuwirken, hat Frau Jana Kalbermatter das Gehäuse eines modernen Audioguides entwickelt und sich ein funktionelles Konzept ausgedacht. Dieser Audioguide, Dojo genannt, soll Kunstobjekte drahtlos erkennen und gespeicherte Informationen darüber via Körperschall zum Museumsbesucher bringen. Dazu muss dieser sich das Ende des stabförmigen Audioguide hinter das Ohr halten. Dank der Informationsübertragung via Körperschall statt mit Kopfhörern, erfüllt der Dojo höchste Hygieneansprüche, was die Benutzung angenehmer macht. Ausserdem soll dieser Audioguide auch als Ticket fungieren und die Zutrittsberechtigung regeln. Das Drücken einer «Like»-Taste soll vermerken, welche Kunstobjekte dem Besucher gefallen haben, damit dieser die Informationen darüber am Ende des Besuchs in Form einer Broschüre oder per E-Mail erhält. Mit diesen Funktionen soll der Dojo den Museumsbesuch hygienisch angenehmer gestalten und helfen, das Allgemeinwissen zu verbessern, damit die Kunstobjekte und nicht der Besuch selbst zu einem unvergesslichen Erlebnis werden. 
Das Ziel des Projekts war das Umsetzen des funktionellen Konzepts von Frau Kalbermatter, indem der Dojo mit Elektronik gefüllt wurde. Dazu wurde die drahtlose Erkennung der Kunstobjekte mittels Bluetooth Low Energy (BLE) Beacons erreicht. Genannte Beacons müssen in unmittelbare Nähe der Kunstobjekte angebracht sein. Die Informationen zu den Kunstobjekten wurden als Audiofiles auf einer herausnehmbaren SD-Karte gespeichert und werden zum Abspielen via PWM-Ausgang des Mikrokontrollers (nRF52832) über einen Klasse D Verstärker auf den Knochenschallaktor gegeben. Tasten für die Wiedergabekontrolle (Start, Stopp, Lauter, Leiser) wurden implementiert sowie die erwähnte «Like»-Taste. Ausserdem verfügt der Dojo über einen Li-Ionen-Akku, welcher induktiv geladen wird. Damit der Dojo gänzlich drahtlos bleibt, erfolgen Datendownload und Konfiguration ebenso über Bluetooth. Der integrierte Mikrokontroller beinhaltet die Software und übernimmt somit die Erkennung, Ansteuerung und Koordination der Hardware. 
Es wurde ein Prototyp der Elektronik realisiert. Die Ansteuerung der Audiofiles erfolgt über Bluetooth-Beacons, welche bis zu einer Distanz von XXX m erkennt werden. Die eingebaute «Like»-Taste ermöglicht, favorisierte Kunstobjekte zu vermerken und die dazugehörigen Informationen am Ende des Besuches digital oder in Form einer Broschüre beim Ausgang als Erinnerung mitzunehmen. Ausserdem besitzt der Dojo einen integrierten Akku mit einer Kapazität von 800 mAh, welcher bei pausenloser Audioausgabe genug Energie für rund 6 Stunden liefert. Die Induktionsladung lädt den Akku zu 100\% innert 13 Stunden. Ausserdem sorgt ein Tiefenentladungsschutz dafür, dass der Dojo bei einer Betriebsnennspannung von unter 3V ausgeschaltet wird um den Akku vor Schäden zu bewahren.
Der nachfolgende Bericht umfasst drei Hauptbereiche. Der erste Bereich (Kapitel 2) umfasst das Gesamtkonzept, welcher die gesamte Anwendung auslegt. Die nachfolgenden zwei Hauptbereiche sind in Hardware (Kapitel 3) und Software (Kapitel 4) gegliedert. Die Hardware teil sich wiederum in die Themengebiete Energieübertragung (Kapitel 3.1), Energiespeicherung (Kapitel 3.2) und Audioausgabe über den Knochenschallaktor (Kapitel 3.5) auf. Die Software beinhaltet die Unterbereiche der State Machine (Kapitel 4.1), Bluetooth (Kapitel 4.3), sowie die gesamte Programmstruktur der SD-Karte (Kapitel 4.4) und Audioausgabe über PWM (Kapitel 4.5). In Kapitel 5 befindet sich die Validierung. 