\subsubsection{LC-Filter}
Vor dem Audioverstärker wird noch ein Filter benötigt, welcher das PWM Signal des Microcontrollers filtert und den DC Anteil entfernt. Um den Microcontroller zu entkoppeln ist ein $100\mu F$ Elektrolytkondensator an jedem PWM Ausgang in Serie geschalten. Das Filter wurde wie folgt dimensioniert. 

\begin{equation}
f_g = \frac{1}{2\cdot \pi \cdot \sqrt{L\cdot C}}
\label{eq LC Filter}
\end{equation}

Um die Gleichung zu lösen wurde eine Grenzfrequenz ($f_g$) von $20kHz$ angenommen und ein Kondensator (C) von $1\mu F$ gewählt. Daraus ergibt sich die Gleichung \ref{eq LC Filter nach L}. 


\begin{equation}
L = \frac{1}{(2\cdot \pi \cdot f_g)^2\cdot C } = 63.33\mu H
\label{eq LC Filter nach L}
\end{equation}

Daraus wurde dann eine L mit dem Wert $68\mu H$ gewählt. Setzt man die Werte in Gleichung \ref{eq LC Filter} ein ergibt sich eine Grenzfrequenz von $19.3kHz$. 
