\subsection{Testkonzept Hardware}\label{sec:testkonzeptHardware}
Damit ein reibungsloser Betrieb möglich ist, müssen die einzelnen Hardware Komponenten auf Herz und Nieren geprüft werden. Nachfolgend werden die Testverfahren von Batterie, induktiver Übertragung und LC-Filter genauer beschrieben und die jeweiligen Testergebnisse aufgelistet.

\subsubsection*{Schutzmechanismen Batterie}\label{sec:batterieSchutzVal}
Die Batterie weist einige integrierte Schutzmechanismen auf, welche alle getestet werden müssen. Als erstes wurde der Tiefentladungsschutz geprüft. Um dies zu testen wurde die Batterie entladen, bis der Schutzmechanismus die Versorgungsspannung abstellt. Da der maximale Entladestrom gleich dem maximalen Ladestrom ist, beträgt dieser gemäss Berechnung \ref{eq:Ladestrom} im Kapitel \ref{sec:energiespeicher} 400mA. Der dazu benötigte Widerstand lässt sich dabei folgendermassen berechnen:
 
\begin{equation}
\centering
R=\frac{U}{I}=\frac{3.7V}{0.4A}= 9.25\Omega
\label{eq:EntladeWiderstand}
\end{equation}

Anschliessend wurde der Entladevorgang mit dem $9.25\Omega$-Widerstand gestartet. Dabei wurde stets die Spannung überwacht, wobei das Absinken von $3.7V$ auf bis $2.5V$ festgestellt wurde. Nach dem die $2.5V$-Schwellenspannung unterschritten wurde, brach der integrierte Batterieschutz die Spannungsversorgung ab. Die Widerstände wurden abgehängt und der gesamte Vorgang wurde kurze Zeit danach mit dem identischen Endergebnis wiederholt.

Als nächstes wurde ein Kurzschlusstest durchgeführt. Der dazu benötigte Schwellenstrom liegt gemäss Datenblatt \cite{LIBattery} bei $4.8A$. Der maximale Widerstand lässt sich dann ebenfalls gemäss Gleichung \ref{eq:EntladeWiderstand} wie folgt berechnen:

\begin{equation}
\centering
R_{max}=\frac{U}{I}=\frac{3.7V}{4.8A}= 0.77\Omega
\label{eq:EntladeWiderstand1}
\end{equation}

Da die berechneten $0.77\omega$ dem maximale Widerstand für die Berechnung entspricht, wurde für die Messung ein Lastwiderstand von $7\Omega$ angeschlossen. Als der Schwellenstrom von $4.8A$ erreicht wurde, riegelte das PCM den hohen Entladungsstrom ab und schaltete die Spannungsversorgung zur Batterie ab. Auch bei diesem Versuch wurde die Messung wiederholt und dabei das gleiche Ergebnis erzielt.

\subsubsection*{Ladeschaltung der Batterie}\label{sec:batterieLadungVal}
Der verwendete Lade-IC hat primär die Aufgabe die Spannung zu regulieren, wobei nach dem Erreichen des Schwellenwertes von $4.2V$, folglich der Strom auf $0A$ herunter reguliert wird. Dieser Vorgang wurde während einem gesamten Ladevorgang der Batterie beobachtet und dokumentiert. Abbildung \ref{fig:LadekurveEmmerich} zeigt die gemessenen Datenpunkte in einem Diagramm. Hierbei ist wichtig zu erwähnen, dass sich dieses Testverfahren nur auf die Funktion des Lade-IC beschränkt und keine induktive Ladevorrichtung angeschlossen wurde. Für die Spannungsversorgung wurde das Netzgerät \glqq Power Supply\grqq\thickspace der Firma \glqq K. Witmer\grqq\thickspace verwendet. Aus diesem Grund ist auch ein Strom von $400mA$ wie auch eine Ladezeit von lediglich rund $270$ Minuten $(\hat{=} 4.5h)$ ersichtlich, was die effektiven Ladewerte mittels induktiver Ladung deutlich unterbietet. Abbildung \ref{fig:LadekurveEmmerich} zeigt die gemessenen Werte in einem Diagramm. Weiter ist die Spannungsregulierung \textcolor{blue}{(blau)} und die Stromregulierung \textcolor{red}{(rot)} zu erkennen. Das theoretische Verhalten der Ladeschaltung ist durch die beiden Kurven gut zu sehen. Nach ungefähr 3 Stunden ist das Einsetzen der Regelung ersichtlich. Sobald die Schwellspannung \textcolor{blue}{(blau)} $4.2V$ erreicht, wird der Ladestrom \textcolor{red}{(rot)} zurückgefahren. Der Ladevorgang wurde beim Erreichen vom $20mA$ unterbrochen.

\begin{figure}[H]
	\begin{center}
		\includegraphics[width=120mm]{data/LadekurveEmmerich.png}
		\caption[Ladekurve Emmerich LI14500]{Ladekurve Emmerich LI14500: Spannungsregulierung und Stromregulierung in Zeitabhängigkeit} %picture caption
		\label{fig:LadekurveEmmerich}
	\end{center}
\end{figure}

\subsubsection*{Induktive Ladeschaltung}\label{sec:batterie}
In diesem Abschnitt werden die Ergebnisse der induktiven Ladeschaltung präsentiert. Nachfolgend zeigt Abbildung \ref{fig:InduzierterStrom} die Abhängigkeit zwischen induziertem Strom und Spannung in Abhängigkeit zur Distanz \glqq z\grqq zwischen den Induktionsspulen.

\begin{figure}[H]
	\begin{center}
		\includegraphics[width=100mm]{data/InduktiveLadung.png}
		\caption[Induzierter Strom in Abhängigkeit der Distanz]{Induzierter Strom in Abhängigkeit der Distanz} %picture caption
		\label{fig:InduzierterStrom}
	\end{center}
\end{figure}

In der Abbildung ist gut sichtbar, dass die Spannung fast linear zur Distanz abnimmt, hingegen der Ladestrom der Batterie extrem schnell abfällt. Aufgrund dieser Erkenntnis sind wir gezwungen eine möglichst kurze Entfernung zwischen den Spulen einzuhalten um eine möglichst effiziente Energieübertragung zu erreichen. Um dies zu erreichen, haben wir eine Ladestation entworfen, welche die bestmöglichen Induktionswerte garantiert. Der minimale Abstand welcher somit während dem Ladezyklus erreicht werken kann ist abhängig von der Wanddicke des Dōjōs. Alles in allem beträgt der Abstand somit rund $1.5mm$. Bei diesem Abstand resultiert ein an die Batterie gelieferter Strom von maximal $70mA$, wobei die Ladezeit direkt von diesem Strom abhängt. Der gesamte Ladezyklus wird in zwei Etappen unterteilt. Bei der ersten Etappe wird die Batterie mit konstantem Strom geladen. Dies hat zur Folge, dass die Spannung von ihrem Minimalwert $3.1V$ auf den Schwellenwert von $4.2V$ reguliert wird. Diese Regulierung wird mit $\Delta U$ in der Rechnung verwendet. Um auf die notwendige Ladezeit des ersten Ladezyklus zu kommen, kann die Zeit ($dt$) während einer beliebig grossen Spannungsdifferenz ($dU$) gestoppt werden. Generell gilt: Umso grösser die Spannungsdifferenz, desto genauer die Approximation. Die Endzeit der Spannungsregelung kann linear hochgerechnet werden. Da jedoch der Ladezyklus nach vollständiger Spannungsregelung noch nicht abgeschlossen ist, folgt noch die benötigte Zeit für die Stromregelung. Da diese nicht einfach berechnet werden kann, wurde dieser Prozess im Labor durchgeführt und gestoppt. Diese zwei Ladezyklen zusammen gerechnet ergeben die gesamte Ladezeit. Für den Ladezyklus wurden drei unterschiedlich hohe Ströme definiert, welche die Zeit für den Ladezyklus vorgeben. Sie beinhaltet einen schonenden Ladezyklus ($I_{1}=30mA$), einen normalen Ladezyklus ($I_{2}=50mA$) und einen schnellen Ladezyklus ($I_{3}=70mA$). Die Erreichung dieser drei Ladestufen wird in dieser Arbeit nicht weiter definiert, könnte jedoch durch drei unterschiedliche Spannungsstufen, welche die Primärspule versorgen, reguliert werden. Nachfolgend sind die Berechnungen für die Ladezeiten bis zum Ende der Spannungsregulierung ($t_{voltage}$) und danach für die Stromregulierung ($t_{current}$) der drei Stufen ersichtlich. Anschliessend wird die gesamte Ladezeit $t_{tot}$ durch die Addition der Teilzeiten berechnet.
\\
\\
\underline{Ladezyklus schonend}
\begin{equation}
\centering
t_{voltage}= \Delta U \cdot \dfrac{dt}{dU}= 1.1V \cdot \dfrac{10min}{6mV} =1'833.3 min= 30h\thickspace 33min
\label{eq:SchonendSpannung}
\end{equation}
\begin{equation}
\centering
t_{current}= 2h\thickspace 7min
\label{eq:SchonendStrom}
\end{equation}

Die gesamte Ladezeit ergibt sich durch Addition der Berechnungen \ref{eq:SchonendSpannung} und \ref{eq:SchonendStrom}.

\begin{equation}
\centering
t_{tot}=t_{{voltage}}+t_{current}=30h\thickspace 33min + 2h\thickspace 7min=32h\thickspace 40min
\label{eq:SchonendGesamt}
\end{equation}

Es ergibt sich somit eine gesamte Ladezeit von knapp $33h$. Dieser Zyklus braucht sehr lange um die Batterie zu laden, er ist aber auch der schonendste. Es wird empfohlen, ihn bei Ruhetagen zu verwenden.
\\
\\
\underline{Ladezyklus normal}
\begin{equation}
\centering
t_{voltage}= \Delta U \cdot \dfrac{dt}{dU}= 1.1V \cdot \dfrac{10min}{16mV} =687.5min= 11h\thickspace 28min
\label{eq:NormalSpannung}
\end{equation}
\begin{equation}
\centering
t_{current}=1h \thickspace 46min
\label{eq:NormalStrom}
\end{equation}

Die gesamte Ladezeit des normalen Ladezyklus ist nachfolgend in \ref{eq:NormalGesamt} ersichtlich.

\begin{equation}
\centering
t_{tot}=t_{voltage}+t_{current}=11h\thickspace 28min + 1h \thickspace 46min=13h\thickspace 14min
\label{eq:NormalGesamt}
\end{equation}
Der normale Ladezyklus benötigt eine Zeit von rund $13h$. Diese Zeit ist schon massiv kürzer als bei der schonenden Ladung. Sie kann für den Ladevorgang während der Nacht oder bei wenigen Besuchern verwendet werden.
\\
\\
\underline{Ladezyklus schnell}
\begin{equation}
\centering
t_{voltage}= \Delta U \cdot \dfrac{dt}{dU}= 1.1V \cdot \dfrac{10min}{26.5mV} =415.1min= 6h\thickspace 55min
\label{eq:SchnellSpannung}
\end{equation}
\begin{equation}
\centering
t_{current}= 1h\thickspace 18min
\label{eq:SchnellStrom}
\end{equation}
Die gesamte Ladezeit gemäss Berechnung \ref{eq:SchnellGesamt} ergibt sich durch Addition der Berechnungen \ref{eq:SchnellSpannung} und \ref{eq:SchnellStrom}.

\begin{equation}
\centering
t_{tot}=t_{voltage}+t_{current}=6h\thickspace 55min + 1h\thickspace 18min = 8h\thickspace 13min
\label{eq:SchnellGesamt}
\end{equation}
Der Ladezyklus mit einem Versorgungsstrom von $70mA$ benötigt eine Ladezeit von knapp $8h$ und ist somit der schnellste Zyklus. Es gilt jedoch zu beachten, dass bei einem Strom von $70mA$ die Spule sich stärker erwärmt als beim Ladezyklus mit $50mA$. Es wird daher empfohlen, dass diese Ladevariante nur bei grossen Besucherzahlen verwendet wird und im Allgemeinen der normale Zyklus mit $50mA$ gestartet wird. Dieser ist zwar um rund $5$ Stunden langsamer, jedoch nachhaltiger und schonender für die eingebauten Materialien.

\subsubsection*{Inbetriebnahme der induktiven Ladeschaltung}\label{sec:erkenntnisse}

Der erste Prototyp der Pulsschaltung für die Induktion wurde mithilfe eines NE555 realisiert, wie sie in Abbildung \ref{fig:NE555} zu sehen ist. Dabei wurden als Widerstände Potentiometer eingebaut, mit welchen die Resonanzfrequenz des LC-Gliedes gesucht werden sollte.Der allererste Versuch ergab eine Übertragung von 5mA bei einer Taktfrequenz von 10kHz. Es stellte sich relativ schnell heraus, dass die Taktfrequenz viel zu tief war. Das konnte durch die Veränderung der Taktfrequenz der Pulsschaltung festgestellt werden. Bei höherer Taktfrequenz wurde die Übertragung besser. Da die Taktfrequenz der NE555-Prototypenschaltung aufgrund der Wahl des Potentiometers auf $10kHz$ begrenzt war, wurde nun in weiteren Schritten mit dem Frequenzgenerator der $2n3055$ Leistungstransistor angesteuert. Des Weiteren wurde das C aus dem LC-Glied ausgebaut, um so das spezifische Maximum der Spule L zu finden. So wurde nun mit konstanter Eingangsspannung und variabler Taktfrequenz die Spule getestet. Mit höher werdender Frequenz wurde die Übertragung besser. Um diese noch weiter zu verbessern empfiehlt sich bei der Taktfrequenz einen Duty Cycle von möglichst $50\%$ zu erzielen. Es erwies sich, dass bei der Taktfrequenz von $93kHz$ die beste Übertragung zustande kam. Bei höheren Frequenzen war diese wieder rückläufig. Da nun die optimale Frequenz gefunden und die Induktivität der Spule bekannt war ($14.9\mu H$), konnte der ideale Kopplungskondensator berechnet werden. Dies ergab, unter Berücksichtigung der E-Reihen, einen $220nF$ Kondensator als ideal. Damit die Spule nicht zu heiss werden und der Isolierlack nicht zu schmelzen beginnt, wurde der Strombegrenzungswiderstand des $2n2222$ mit $2\Omega$ definiert. Die Berechnung hierzu ist im oberen Text zu finden. Somit sind alle benötigten Bauteile der Schaltung aus Abbildung \ref{fig:NE555} bekannt.

\begin{figure}[H]
	\begin{center}
		\includegraphics[width=80mm]{data/Ne555circuit.png}
		\caption[Verwendete Timerschaltung NE555 als Pulsquelle]{Verwendete Timerschaltung NE555 als Pulsquelle} %picture caption
		\label{fig:NE555}
	\end{center}
\end{figure}

$C = 1nF$\\
$R_{1} = 510\Omega$\\
$R_{2} = 7.5 k\Omega$\\

Mit diesen Werten konnte nach dem Gleichrichter, bei einer Übertragungsdistanz von einer Standard-Platinen dicke, eine Leerlaufspannung von ca. $17V$ und ein Kurzschlussstrom ca. $300mA$ gemessen werden. Es gilt jedoch zu beachten, dass diese Werte ohne Last gemessen wurden. 

Da nun die Induktionsstufe funktioniert wird diese nun um die Ladeschaltung erweitert. Hier ergab sich folgendes Problem:

Die Eingangsspannung des Lade-IC’s darf nicht mehr als $6.5$ Volt betragen. Da die Receiverspule sich im inneren des Dōjōs befindet und dadurch nur eine sehr geringe Baugrösse besitzen darf, kann diese zwangsläufig keine grossen Leistungen erbringen. Dies bedeutet, dass akzeptabel Ladeströme nur mit genügend hohen Spannungen erzeugt werden können.

Um dies zu erzielen wurde versucht, einen Spannungsregler einzubauen, um den übertragenen Strom beizubehalten, jedoch die Spannung zu begrenzen. Das Ergebnis davon war jedoch ziemlich bescheiden. Der so entstehende Energieverlust ist ziemlich gross und ausserdem kam es vereinzelt vor, dass der Lade-IC trotzdem kaputt ging, da der Spannungsregler nicht sauber regelte. 

Die Ideale Lösung wäre natürlich einen geeigneteren Lade-IC zu nehmen. Da jedoch die Validierung desjenigen bereits durchgeführt wurde und nicht genügend Zeit vorhanden war diesen erneut zu bestellen wurde sich entschieden die Transceiver Schaltung ein wenig anzupassen.

Dabei gibt es folgende Möglichkeiten. Man kann die Eingangsspannung absenken. Dies beeinflusst jedoch auch den Übertragenen Strom exponentiell, da einerseits eingangsseitig mit weniger Spannung auch weniger Strom durch die Spule fliesst und andererseits auch der Leistungsstransistor nicht sauber durchgesteuert wird da die Pulsschaltung ebenfalls an der gleichen Versorgungsspannung hängt und somit sein Pulssignal beeinflusst wird. Die zweite Möglichkeit ist die Taktfrequenz verringern. Dadurch wird die Induzierte Spannung ebenfalls abgesengt was ebenfalls den Strom exponentiell beeinflusst. Die dritte Möglichkeit wäre die Übertragungsdistanz zu verringern. Da aber sichergestellt werden wollte, dass der Dōjō-Boden dennoch eine gewisse Stabilität aufweisen sollte wurde diese Distanz der Platinendicke beigehalten.

Somit wurde die Schaltung lange ausgetestet um die beste Kombination zu erzielen. Das beste Resultat ergab die Kombination zwischen einer Taktfrequenz von ca 80kHz und einer Eingangs Spannung von $8.3V$. Dies lässt sich damit erklären das leichtes Absenken der beiden Möglichkeiten den Strom nicht so stark beeinflusst. Des Weiteren wurde der Strombegrenzungswiderstand auf $1.5\Omega$ verringert da sich die Eingangs Spannung verringert hat und ausserdem der Stromfluss durch die Spule damit erhöht wurde. 

Diese Anpassungen fielen ebenfalls zu Gunsten des Energiemanagements, jedoch ist dies eine stabile Variante für die ersten Prototypen. Durch die Anpassungen wird an der Transceiver Spule die überschüssige Energie in Form von Wärme  abgegeben. Die Tests ergaben, dass so die Batterie bei einer Spannung von 4.7V mit 70mA geladen werden könne. Es ist ebenfalls möglich kurzzeitig mit einem höheren Strom zu laden, wenn die Eingangs Spannung nur geringfügig erhöht wird. (bis zu 120mA). Jedoch ist die Hitzeentwicklung hierbei so gross, dass nach einer gewissen Zeit die Übertragung zusammenbricht. Deshalb ist zu empfehlen die angegebenen Werte nicht zu überschreiten um eine dauernde Funktionalität zu gewährleisten.

Die Werte der oben beschriebenen angepassten Pulsschaltung werden ebenfalls wie in Abbildung \ref{fig:NE555} aufgebaut und sind folgende:\\
C: 1nF\\
R1: 200 $\Omega$\\
R2:9 k$\Omega$\\


\subsubsection*{LC-Filter} \label{sec:Validierung LC-Filter}
Die Validierung des LC-Filters wurde mit einem $500Hz$ Sinus-Signal durchgeführt. Dabei wurde sowohl ohne Filter wie auch mit Filter ein KO-Bild aufgenommen. Abbildung \ref{fig:filterVal} zeigt das Verhalten des Filters mit und ohne Filterstufe. Ebenfalls ist die Fast Fourier Transform (FFT) in \textcolor{red}{rot} zu erkennen. Sie gibt eine Auskunft über das enthaltene Frequenzspektrum.

\begin{figure}[htbp]
	\centering
	\subfigure[Aufnahme ohne Filterstufe]{\includegraphics[width = 0.48\textwidth]{data/500Hz_Sin_ohne_Filter_FFT.png}}\quad
	\subfigure[Aufnahme mit Filterstufe]{\includegraphics[width = 0.48\textwidth]{data/500Hz_Sin_mit_Filter_FFT.png}}\quad
	\caption[Aufnahme der Filterstufe mit 500Hz Sinus-Signal]{Aufnahme der Filterstufe: links ohne Filterstufe und rechts mit Filterstufe}
	\label{fig:filterVal}
\end{figure}

Durch die beiden KO Aufnahmen wurde ersichtlich, dass durch den Filter ein deutlich besseres und weniger verrauschtes Signal erzielt wird. Somit ist der LC-Filter notwendig, da mit ihm die Schaltfrequenzen des PWM-Ausgangs unterdrückt werden.
