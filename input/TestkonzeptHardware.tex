\subsection{Testkonzept}
Damit ein reibungsloser Betrieb möglich ist, müssen die einzelnen Hardware Komponenten auf Herz und Nieren geprüft werden. Nachfolgend werden die Testverfahren genauer beschrieben und die Testergebnisse aufgelistet.

\subsubsection{Batterie}
Die Batterie weist einige Schutzmechanismen auf, welche alle getestet werden müssen. Als erstes wurde der Tiefentladungsschutz geprüft. Um dies zu testen wurde ein Winderstand der Dimension 9$\Omega$ angeschlossen, wobei gemäss Berechnung \ref{eq:Entladestrom} ein Entladestrom von rund 400mA resultierte.

\begin{equation}
\centering
I_{entlade}=\frac{U}{R}=\frac{3.7V}{9\Omega }= 411mA
\label{eq:Entladestrom}
\end{equation}

Während dem Entladevorgang wurde stets die Spannung überwacht, wobei die Spannung von 3.7V auf bis 2.5V absank. Nach dem die 2.5V Schwellenspannung unterschritten wurde, brach der integrierte Batterieschutz die Spannungsversorgung ab. Die Widerstände wurden abgehängt und der gesamte Vorgang wurde mit Erfolg wiederholt.
\newline
Als nächstes wurde ein Kurzschlusstest durchgeführt, wobei hier der Schwellenstrom gemäss Datenblatt bei 4.8 liegt. Gemäss dem U=R$\cdot$I Gesetz, wurde ein Widerstand der Grösse von 700m$\Omega$ verwendet damit der Grenzwert überschritten wird. Auch bei diesem Versuch, riegelte das PCM den hohen Entladungsstrom ab und schaltete die Versorgungsspannung ab.

