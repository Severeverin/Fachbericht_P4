\section*{Abstract}\label{sec:abstract}

Aimless wandering and too little knowledge are the reasons why the visitation of a museum is more memorable than its artworks. Because of that, an audio guide called Dōjō was designed to help make the artworks unforgettable.

The goal of this project was to design the circuits for the Dōjō. The Dōjō should be able to recognise the artwork the visitor is standing in front as well as playing audio information over a bone sound sensor. In addition to that, the visitor should be able to “like” an artwork by pressing a button and get the information of the liked artworks at the end of his visit by email or printing.

The Dōjō gives the information to the visitor by the bone sound sensor. For that, the sound files must be stored on a SD-Card which is placed in the device. Each artwork has a Bluetooth Low Energy (BLE) Beacon that sends its ID to the Dōjō. The Dōjō’s internal microcontroller is scanning for BLE-Signals and plays the file that is belonging to the ID of the strongest received signal, which is assumed to be coming from the nearest beacon. Furthermore, the Dōjō has an inductive rechargeable battery.

The correct beacon is identified at XXX meters. The bone sound sensor plays the audio files with a loudness of XXX dB and no distortion. The inductive rechargeable battery with a capacity of XXX mAh can provide power for XXX hours, full featured and is recharged after XXX hours of charging. The deep discharge protection turns off the Dōjō when the battery is at the XXX $\%$ of its power to protect the battery from damage.

The Dōjō can be improved by implementing wireless data transfer and room access authorization. Nevertheless, it's a significant win for museum visitors.

Key Words: audio guide, Bluetooth low energy, inductive charge, bone sound sensor
