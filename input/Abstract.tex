\section*{Abstract}\label{sec:abstract}
Aimlessly wandering around a museum because of too little knowledge is one of the main negative reactions to a museum visit. That’s why no one can remember an individual work of art. To counteract this, an audio guide called Dōjō was designed to provide visitors with key information about a work of art so they can enjoy their visit more. This project should design the circuits for this Dōjō, which electronically recognises the artwork the visitor is standing in front of. Audio information stored on an internal SD-Card over a Class-D Amplifier is provided through an integrated bone sound sensor produced by adafruit. Buttons for audio control have also been integrated. In addition, the Dōjō allows the visitor to \glqq like\grqq an artwork by simply pressing a button. Information about these \glqq liked\grqq works are made available to the visitors either by email or a hardcopy is provided at the end of their visit. Each work of art has a Bluetooth Low Energy (BLE) Beacon that continuously sends its ID to the Dōjō. The Dōjō’s internal microcontroller (nrf52832) then scans for this BLE-Signals and plays the file that belongs to the ID with the strongest signal. The Dōjō can identify beacons at XX meters. The inductive rechargeable battery has a capacity of 800mAh and can provide energy for about 6 hours when the Dojo is playing audiofiles and be recharged in normal charging mode within 13 hours. To protect the battery from damage, a deep discharge protection turns off the Dōjō when the battery is under 3V of its nominal voltage.\\[0.25cm]
Key Words: audio guide, Bluetooth low energy, inductive charge, bone sound sensor