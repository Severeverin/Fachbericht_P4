\subsection{Verstärkerstufe} \label{sec:verstaerkerstufe}
Nachdem die Steuerungseinheit gezeigt wurde, kann nun die Ausgangsstufe erläutert werden. Mit einer Verstärkerstufe lassen sich auf einfache Art und Weise Signale jeglicher Form verstärken. Sie eignen sich bestens, um den Ausgang eines Mikrocontrollers entsprechend aufzubereiten, da die Ausgangsseite meist sehr niedrige Ströme aufweist. Dadurch kann dem Knochenschallaktor genügend Energie zur Verfügung gestellt werden. Prinzipiell gibt es zwei Arten von Verstärkern. Entweder erfolgt die Umsetzung digital oder analog. Beide erfüllen die gleiche Aufgabe, weisen jedoch bezüglich Wirkungsgrad einen deutlichen Unterschied auf. Die digitale Variante weist ungefähr einen Wirkungsgrad von 90$\%$ auf \cite{BoneConductorAdafruit}, während die analoge Variante einen maximalen Wirkungsgrad im Bereich der Leistungsanpassung erzielt \cite{Niklaus_Skript}. Aus diesem Grund wird ein digitaler Verstärker (Class-D-Verstärker) in der Anwendung implementiert. Die Wahl fiel auf den Stereo-Amplifier MAX 98306. Der Verstärker hat einen Stromverbrauch von $143mA$ und eine Speisespannung von $3.3V$. Somit hat er einen Leistungsverbrauch von $471.9 mW$. Im Standby benötigt er lediglich $2 mA$ und somit $6.6 mW$.