\subsection{Verstärkerstufe} \label{sec:verstaerkerstufe}

Mit einer Verstärkerstufe lassen sich auf einfache Art und Weise Signale jeglicher Form verstärken. Sie eignen sich bestens, um den Ausgang eines Mikrocontrollers entsprechend aufzubereiten, da die Ausgangsseite meist sehr niedrige Ströme aufweist. Dadurch kann dem Knochenschallaktor genügend Energie zur Verfügung gestellt werden.

Prinzipiell gibt es 2 Arten von Verstärkern. Sie können digital oder auch analog umgesetzt werden. Beide erfüllen die gleiche Aufgabe, weisen jedoch bezüglich Wirkungsgrad einen deutlichen Unterschied auf. Die digitale Variante weist ungefähr einen Wirkungsgrad von 90$\%$ auf \cite{BoneConductorAdafruit}, während die analoge Variante maximal einen Wirkungsgrad im Bereich der Leistungsanpassung erzielt \cite{Niklaus_Skript}. Aus diesem Grund wird ein digitaler Verstärker (Class-D-Verstärker) in der Anwendung implementiert. die Wahl fiel auf den Stereo-Amplifier MAX 98306 (nicht ganz sicher was nacher wirklich implementiert wird).

(Strom/Spannung/Leistung) ergänzen und begründen, Berechnung der Verstärkung